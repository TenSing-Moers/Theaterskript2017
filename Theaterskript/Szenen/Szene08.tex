% This file is part of TenSing-Moers/Theaterskript2017.
%
% TenSing-Moers/Theaterskript2017 is free content: you can redistribute and/or
% modify it under the terms of the cc-by-nc-sa (Creative Commons
% Attribution-NonCommercial-ShareAlike) as released by the
% Creative Commons organisation, version 4.0.
%
% TenSing-Moers/Theaterskript2017 is distributed in the hope that it will be useful,
% but without any warranty.
%
% You should have received a copy of the cc-by-nc-sa-license along
% with this copy of TenSing-Moers/Theaterkskript2017. If not, see
% <https://creativecommons.org/licenses/by-nc-sa/4.0/legalcode>.
%
% Copyright TenSing Moers and all whose work and <3 went in this project.
\Scene{Der große Plan}
\DisplayPersons
%\chars{Helge Schneider}
\ort{Unterwegs zur Schokoladenfabrik}
\requ{Bogen}

\StageDir{\remyh, \olafh, \thomash, \schneewittchenh, \doch, \jackh, \meridah und \aladdinh sind auf dem Weg zur Schokoladenfabrik und beratschlagen sich unterwegs. Sie laufen auf die Bühne}
\mikroan{1,2,3,4,5,6}

\begin{drama}
\olafx{Ohh, da vorne ist die Schokoladenfabrik! Mhh\dots\ Schokolade\dots\ \direct{will loslaufen}}
\remyx{\direct{hält ihn zurück} Nein \olaf! Hast du denn nicht zugehört? Wir müssen uns gegen die Fabrik wehren!}
\jackx{Und was gedenkst du, zu tun?}
\meridax{Ich finde, wir sollten uns einfach alle bewaffnen und die Fabrik stürmen! Wird Zeit, dass wir mal wieder was unternehmen!}
\aladdinx{Im Ernst? Das ist die dümmste Idee, die ich je gehört habe! Wir müssen uns anschleichen und über die Mauer klettern!}
\meridax{Also, das ist doch nur noch dümmer. Was willst du eigentlich, du Möchtegern-Prinz?}
\aladdinx{Möchtegern-Prinz? Also das ist ja wohl die Höhe. 'Ne ordentliche Prinzessin bist du ja wohl auch nicht!}
\meridax{Ich bin Merida, Erstgeborene des Clans DunBronch, und ich lasse mir von niemandem erzählen, was ich zu tun und zu lassen habe! Mein Plan ist besser!}
\aladdinx{Das wollen wir ja sehen.}
\end{drama}

\StageDir{\meridah und \aladdinh machen sich startklar, \meridah untersucht ihren Bogen}

\begin{drama}
\remyx{Halt, stopp! \direct{geht dazwischen} Ihr verratet uns noch alle!}
\olafx{\direct{traurig} Ja, und dann gibt es keine Schokolade.}
\thomasx{Wir sollten erstmal beratschlagen und uns einen guten Plan überlegen!}
\schneewittchenx{Ja, er hat Recht. Ihr seid so kindisch! Was meinst du, \doch?}
\docx{Also, meine Liebste, ich denke, dass ich vermutlich in der Lage wäre, eventuell, nach genauerem Beobachten, einen Plan zu entwickeln, der uns vielleicht\dots}
\jackx{Was für ein wirres Gerede! Ich finde, wir sollten erstmal\dots}
\remyx{Halt, ich habe eine Idee!}
\end{drama}

\StageDir{Alle stellen sich in einem Kreis um \remyh und tuscheln}

\begin{drama}
\olafx{Oh, das klingt toll!}
\schneewittchenx{Das könnte funktionieren.}
\aladdinx{Also ich fand meinen Plan zwar besser, aber\dots}
\meridax{Typisch! Schon wieder bist DU natürlich viel besser!}
\remyx{Aufhören, alle beide! Ihr seid unmöglich. Besser, ihr geht zurück ins Versteck, Streithähne können wir bei unserer Mission nicht gebrauchen.}
\thomasx{Ich glaube auch, dass es so sicherer ist für uns alle. Gleich darf nichts schiefgehen! Und jemand muss schließlich auch \pocahontas über unseren Plan informieren.}
\remyx{Also gut. \direct{zu den anderen} Ihr kennt den Plan. Ich gehe mit den anderen zurück, sonst verraten sie uns noch.}
\end{drama}

\StageDir{\thomash, \doch, \olafh, \jackh und \schneewittchenh gehen von der Bühne}

\begin{drama}
\schneewittchenx{\direct{im Gehen} Sowas von kindisch, ich sag's euch!}
\mikroaus{1,2,3}
\remyx{\direct{zu \aladdin und \merida} Also ich gebe \schneewittchen ja nicht oft Recht, aber eure Streitereien müssen aufhören.}
\meridax{\aladdin hat angefangen!}
\aladdinx{Ist das dein Ernst? Du hast doch\dots}
\end{drama}

\mikroaus{4,5,6}
\StageDir{\remyh macht auf dem Absatz kehrt und geht von der Bühne. \aladdinh und \meridah halten Inne, und folgen ihm dann. Licht aus.}
