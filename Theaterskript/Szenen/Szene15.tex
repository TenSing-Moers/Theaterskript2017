% This file is part of TenSing-Moers/Theaterskript2017.
%
% TenSing-Moers/Theaterskript2017 is free content: you can redistribute and/or
% modify it under the terms of the cc-by-nc-sa (Creative Commons
% Attribution-NonCommercial-ShareAlike) as released by the
% Creative Commons organisation, version 4.0.
%
% TenSing-Moers/Theaterskript2017 is distributed in the hope that it will be useful,
% but without any warranty.
%
% You should have received a copy of the cc-by-nc-sa-license along
% with this copy of TenSing-Moers/Theaterkskript2017. If not, see
% <https://creativecommons.org/licenses/by-nc-sa/4.0/legalcode>.
%
% Copyright TenSing Moers and all whose work and <3 went in this project.
\renewcommand{\sechl}{\tikz[overlay]{ \draw[fill=Orange1] (-.6\textwidth,-1.2em) rectangle (1.15\textwidth,1em);}}%\draw (-.75\textwidth,-1ex) -- (\textwidth,-1ex); \draw (-\textwidth,1em) -- (\textwidth,1em)}%}
\Scene{Und wenn sie nicht gestorben sind\dots}
\renewcommand{\sechl}{\tikz[overlay]{ \draw[fill=Orange1] (-.6\textwidth,-1ex) rectangle (1.15\textwidth,1em);}}%\draw (-.75\textwidth,-1ex) -- (\textwidth,-1ex); \draw (-\textwidth,1em) -- (\textwidth,1em)}%}
\DisplayPersons
%\chars{Helge Schneider}
\ort{Mitsibitsi-Elektro-Halle}
\requ{Eine Banane}
\mikroan{1,2,3,4,5,6}

\begin{drama}
\managerax{\direct{zu \thomas} Danke für ihre große Hilfe! Ohne Sie hätten wir wahrscheinlich nie eine Lösung gefunden. Und wer weiß, wozu die \rk \mickeyy sonst noch alles überredet hätte.}
\mickeyx{Mir tut es so leid, was ich getan habe. Wie konnte ich nur so egoistisch sein? Wie konnte ich mich nur so manipulieren lassen? Ich war einfach so traurig, dass ihr mich nicht mehr haben wolltet.}
\remyx{Aber \mickeyy, du weißt doch, dass wir eine große Familie sind. Wir lassen dich niemals im Stich!}
\thomasx{Ich habe euch auch gerne geholfen. Ich habe viel neues gelernt! Und \mickeyy, jeder macht mal Fehler, das kommt vor!}
\aladdinx{Genau. Das wichtigste ist, dass man seine Fehler erkennt und nicht wiederholt!}
\meridax{Da bin ich ausnahmsweise mal deiner Meinung!}
\schneewittchenx{\direct{schaut in ihren Schminkspiegel} Also ich mache ja nie Fehler\dots}
\end{drama}
 
\StageDir{Alle schauen \schneewittchen vorwurfsvoll an}

\begin{drama}
\managerex{So, Herr \gottschalk, jetzt wird's aber langsam mal Zeit!}
\managerax{Es ist wohl der Moment für sie gekommen, nach Hause zu gehen.}
\thomasx{Da habt ihr wohl Recht. \krawatte Es war eine sehr schöne Zeit, ich freue mich sehr, euch alle kennengelernt zu haben!}
\olafx{\direct{traurig} Aber du schreibst uns doch eine Karte, oder?}
\thomasx{Natürlich! Ich werde euch auch ganz schön vermissen.}
\customx{Alle}{Wir dich auch!}
\olafx{\direct{Umarmt \thomas} Ohh}
\end{drama}

\StageDir{Alle umarmen sich und frieren ein. Die \rkh kommt auf die Bühne, sieht die Gruppe und verzieht das Gesicht}

\begin{drama}
\rkx{Das geht mir zu weit. Ab mit ihren Köpfen! Ich gehe woanders hin.}
\end{drama}

\StageDir{\rkh verlässt die Bühne wieder, die Handlung geht weiter}

\begin{drama}
\managerbx{Herr \gottschalk, sind sie bereit? Wir werden sie jetzt in ihre Heimat zurückbeamen.}
\thomasx{Ja, so bereit wie noch nie! Ich freue mich auf meine Goldbären\dots}
\olafx{Gute Reise, und bis bald hoffentlich!}
\end{drama}

\StageDir{Die Manager stellen sich in einen Kreis um \thomash und murmeln Zaubersprüche, es blitzt und qualmt}

\begin{drama}
\thomasx{Auf Wiedersehen! Ihr werdet mir fehlen. Aber eins sage ich euch!}
\end{drama}

\StageDir{Qualm und Lichtblitze werden stärker}

\begin{drama}
\thomasx{Nach diesen Geschehnissen werde ich ganz bestimmt nicht mehr Wetten, dass...?! moderieren!}
\end{drama}

\mikroaus{1,2,3,4,5,6}
\StageDir{Knall, Licht aus, Ende!}