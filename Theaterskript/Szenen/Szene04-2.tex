% This file is part of TenSing-Moers/Theaterskript2017.
%
% TenSing-Moers/Theaterskript2017 is free content: you can redistribute and/or
% modify it under the terms of the cc-by-nc-sa (Creative Commons
% Attribution-NonCommercial-ShareAlike) as released by the
% Creative Commons organisation, version 4.0.
%
% TenSing-Moers/Theaterskript2017 is distributed in the hope that it will be useful,
% but without any warranty.
%
% You should have received a copy of the cc-by-nc-sa-license along
% with this copy of TenSing-Moers/Theaterkskript2017. If not, see
% <https://creativecommons.org/licenses/by-nc-sa/4.0/legalcode>.
%
% Copyright TenSing Moers and all whose work and <3 went in this project.
\section{Der Neue}
%\chars{Helge Schneider}
\ort{Im Geheimversteck}
\requ{Eine Banane}

\StageDir{\remyh, \schneewittchenh, \doch und \thomash stürmen in's Geheimversteck (auf die Bühne). \olafh, \aladdinh, \jackh und \meridah sitzen auf dem Boden und spielen Karten.}
\mikroan{1,2,3,4,5,6}

\begin{drama}
\schneewittchenx{\direct{ruft laut} Ihr GLAUBT nicht, was gerade passiert ist!}
\end{drama}

\StageDir{\olafh, \aladdinh, \jackh und \meridah schauen erstaunt auf, legen die Karten ab und stehen auf}

\begin{drama}
\meridax{\direct{Fasst sich als erste} Da seid ihr ja endlich! Wir haben uns schon Sorgen gemacht!}
\olafx{Ja, genau! Schön, dass ihr wieder da seid! \direct{Umarmt \doch, der sofort ersteift und sich der Umarmung zu entziehen versucht} Aber wer ist denn der neue? \direct{Geht zu \thomash und stupst ihn an}}
\remyx{Das ist \thomas! Wir haben ihn beim Casting aufgegabelt.}
\docx{Ja, er war auf einmal da und wusste nicht, wo er sich befindet. Aber viel wichtiger ist: Mickey ist \textit{tatsächlich} beim Casting aufgetaucht! Und nicht nur irgendwie, nein, er hat den Raum gestürmt und versucht, uns alle zu hypnotisieren! Wir konnten gerade noch rechtzeitig fliehen, und weil wir es nicht besser wussten, haben wir \thomass mitgenommen.}
\jackx{Aha, das klingt ja nach vorzüüüglicher Unterhaltung. Und wie gedenkt ihr, jetzt weiter fortzufahren?}
\meridax{Vielleicht erzählen wir \thomass erstmal alles. Er sieht nach wie vor ziemlich verwirrt aus. Also, es hat alles damit angefangen, dass\dots}
\aladdinx{\direct{Unterbricht \merida besserwisserisch und provokant} Stopp, stopp, stopp! Ich denke, dass ich das wohl besser erklären kann als du!}
\meridax{\direct{genervt, zickig} \aladdin! Lass' mich doch einmal ausreden, und misch' dich nicht immer in alles ein!}
\aladdinx{Ich mische mich nicht ein! Es ist nunmal so, dass\dots}
\jackx{\direct{Unterbricht ihn} Kinder, Kinder\dots\ Nun streitet euch doch nicht! Ich denke, ich sollte es ihm erklären. \direct{Wendet sich zu \thomas} Also, \thomass, es ist ganz einfach: \mickey soll nicht mehr länger unser Maskottchen sein, und deshalb will er sich an Disney rächen. Verstanden?}
\thomasx{Bitte was? Um ehrlich zu sein, nein, ich habe keinen Schimmer! Muss ich das verstehen?}
\remyx{Ach, war doch klar, dass das nichts wird. Pass auf, ich erkläre es dir! Komm, setz dich am Besten, es könnte etwas länger dauern.}
\end{drama}

\StageDir{Alle setzen sich hin.}

\begin{drama}
\remyx{Kennst du \mickey Mouse?}
\thomasx{Ja klar. Wer kennt denn \mickey Mouse schon nicht?}
\remyx{Gut. Dann kennst du sicherlich auch das Disneyland\dots}
\schneewittchenx{\direct{Unterbricht ihn schnippisch} Er ist doch gerade hier, natürlich kennt er es!}
\thomasx{Disneyland? Achso, ich bin in Paris\dots\ Oder in Kalifornien?}
\aladdinx{Was ist Kalifornien?}
\meridax{Also das hier jedenfalls nicht!}
\remyx{Nein, was auch immer das für Orte sind, dort bist du nicht. Du bist im Disneyland! In unserer Heimat!}
\thomasx{Aber das ist doch\dots\ \direct{Richtet nervös seine Krawatte} Ich dachte, ich wäre von diesem \ron in's Disneyland nach Paris oder so gebeamt worden\dots}
\schneewittchenx{Ich habe doch gleich gesagt, er kennt Disneyland nicht. Wenn man solche Klamotten trägt, ist das ja wohl zu erwarten!}
\jackx{Wenn ich darauf hinweisen könnte, Mylady, eigentlich sagten sie das exakte Gegenteil davon\dots}
\remyx{\direct{unterbricht ihn} Nein, Stopp, nicht schon wieder! Also gut, der Reihe nach. Disney wurde seit jeher von \mickey repräsentiert. Aber die Manager, die sich hier um alles kümmern, haben beschlossen, dass \mickey zu alt und langweilig geworden ist, um weiterhin unser aller Maskottchen zu sein. Auf dem Casting sollte heute der Nachfolger gewählt werden. Das gefällt \mickey natürlich gar nicht, und deshalb versucht er sich, an ganz Disney zu rächen. Klar so weit?}
\thomasx{Bei dieser Castingshow\dots\ Das mit der Waffe war wirklich \mickey Mouse? Also, ich meine\dots\ DER \mickey Mouse? Wieso hat er alle bedroht? Und was waren das für komische Lichtblitze, und wieso sind wir abgehauen?}
\remyx{Nun, \mickey ist natürlich sehr wütend darüber, dass er einfach abgesetzt wurde. Er will jetzt alles unter seine Kontrolle bringen, um sich zu rächen. Das Licht war eine Hypnose, und wir sind geflohen, um ihm nicht hilflos ausgeliefert zu sein, so wie alle anderen. Nun ist es an uns, Disneyland zu retten! Wir müssen \mickey aufhalten!}
\thomasx{\mickey versucht Disneyland unter seine Kontrolle zu bringen? Aber woher wusstet ihr das?}
\docx{Nun, es war zu erwarten. Wir haben\dots}
\jackx{Die \rk hat den traurigen \mickey dazu gebracht, seine Trauer in Zorn umzuwandeln und unser schnuckeliges Land zu unterwerfen!}
\meridax{Aber wir sind die letzte Hoffnung, und werden kämpfen bis zum Sieg! \direct{Hebt die Faust in die Luft}}
\remyx{Ehm ja, so in etwa. Wir haben davon erfahren, dass \mickey irgendwas vorhat, und uns nach und nach zusammengeschlossen, um das zu verhindern. Zu dem Casting sind wir gegangen, weil wir mehr Informationen brauchen. Wir mussten wissen, wie \mickey vorgehen wird, und es war zu erwarten, dass sie sich das Casting nicht entgehen lassen wird. Jetzt wissen wir es, Hypnose ist ihr Mittel.}
\thomasx{Okay, ich verstehe\dots\ Das klingt ja alles grauenvoll! Disneyland ist so ein schöner Ort voll Freude und Glück, der soll einfach zerstört werden? Ich werde euch dabei helfen, gegen \mickey vorzugehen! Zumindest solange, bis ich herausgefunden habe, wie ich wieder nach Hause komme\dots\ \direct{Zu sich selbst} Auch wenn ich mir immer noch nicht sicher bin, ob ich das hier nicht alles träume\dots}
\remyx{Träumen? Wieso solltest du träumen? Ich mache uns erstmal was zu essen. \merida, \doc, \olaf, geht ihr bitte ein paar Pilze sammeln? Und nehmt \pocahontas mit, die braucht mal ein bisschen Gesellschaft. In letzter Zeit ist sie schon sehr merkwürdig geworden\dots}
\meridax{Sie war doch schon immer merkwürdig! Als ob sich daran jemals etwas ändern würde\dots\ Aber gut, wir versuchen es. Bis gleich!}
\end{drama}

\mikroaus{1,2,3,4,5,6}
\StageDir{\meridah, \doch und \olafh verlassen die Bühne. Licht aus.}