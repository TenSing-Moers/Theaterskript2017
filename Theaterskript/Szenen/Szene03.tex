% This file is part of TenSing-Moers/Theaterskript2017.
%
% TenSing-Moers/Theaterskript2017 is free content: you can redistribute and/or
% modify it under the terms of the cc-by-nc-sa (Creative Commons
% Attribution-NonCommercial-ShareAlike) as released by the
% Creative Commons organisation, version 4.0.
%
% TenSing-Moers/Theaterskript2017 is distributed in the hope that it will be useful,
% but without any warranty.
%
% You should have received a copy of the cc-by-nc-sa-license along
% with this copy of TenSing-Moers/Theaterkskript2017. If not, see
% <https://creativecommons.org/licenses/by-nc-sa/4.0/legalcode>.
%
% Copyright TenSing Moers and all whose work and <3 went in this project.
\Scene{Auf der Flucht}
\DisplayPersons
%\chars{\thomash \delim \remyh \delim \schneewittchenh \delim \doch}
\ort{Irgendwo auf dem Weg zum Geheimversteck}
\requ{Eine Banane}

\StageDir{\thomash, \remyh, \doch und \schneewittchenh stürmen auf die Bühne.}

\begin{drama}
\thomasx{\direct{Außer Atem} He, wartet mal bitte!}
\end{drama}

\StageDir{Die vier bleiben stehen.}

\begin{drama}
\schneewittchenx{Was ist denn los, wir müssen weiter!}
\thomasx{Nein, im Ernst, wartet mal bitte. Wo bin ich hier eigentlich? Wer seid ihr? Und was war das da gerade?}
\remyx{Du bist in Disneyland. Ich bin \remy, vielleicht kennst du mich ja? Das sind \schneewittchen und \doc.}
\thomasx{\direct{verwundert} \remy? Disneyland? Und \schneewittchen? Märchenschlösser und sprechende Ratten? Vielleicht hat dieser \ron ja doch nicht nur Quatsch erzählt...}
\docx{Wer ist \ron?}
\schneewittchenx{\direct{Unterbricht das Gespräch} Hey, stopp. Seid ihr euch sicher, dass wir überhaupt mit ihm reden sollten? \remy, \doc, kommt mal her!}
\end{drama}

\StageDir{\remyh und \doch gehen zu \schneewittchenh, die drei stellen sich in einen Kreis etwas abseits von \thomas. \thomash schaut sich fragend um, während sich die drei beraten.}

\begin{drama}
\remyx{Du denkst, er ist ein Spion? Aber er weiß doch noch nicht einmal, wo er hier ist! Und außerdem, was sollen wir denn machen? Wenn wir ihn einfach hier lassen, dauert es keine 10 Minuten, bis einer von \mickey's Schergen ihn findet...}
\schneewittchenx{\direct{leise} Aber schaut euch doch mal an, wie er angezogen ist! So jemandem kann man doch gar nicht trauen!}
\remyx{Ich finde, wir sollten ihm eine Chance geben.}
\docx{Ja, also, ich meine \dots}
\schneewittchenx{\direct{unterbricht ihn} \doc, findest du nicht auch, dass dieser Anzug absolut inakzeptabel ist?}
\docx{Das mag sein, mein Schatz aber darauf kommt es doch gerade gar nicht an. Ich finde...}
\thomasx{\direct{ruft zu den dreien herüber, unterbricht sie} He, was ist denn nun? Erst hattet ihr es so eilig, und jetzt quatscht ihr stundenlang rum? Das ist doch sonst immer meine Aufgabe!}
\remyx{Er hat Recht, sollen wir ihn nicht erstmal fragen, wie er hier hergekommen ist?}
\docx{Das wollte ich gerade vorschlagen!}
\end{drama}

\StageDir{\schneewittchenh, \doch und \remyh gehen wieder zu \thomas. \schneewittchenh ergreift das Wort.}

\begin{drama}
\schneewittchenx{Woher hast du eigentlich diesen scheußlichen Anzug?}
\end{drama}

\StageDir{\remyh stößt sie energisch an}

\begin{drama}
\schneewittchenx{Ähm... Ich meine... Wie bist du eigentlich hier hergekommen?}
\thomasx{Das... Weiß ich selbst nicht so genau. Aber ich werde euch alles erzählen, was ich glaube, zu wissen.}
\docx{Schieß los!}
\thomasx{Also, in meinem Land moderiere ich eine Show, bei der Kandidaten versuchen, Talent und Mut zu beweisen, indem sie verrückte Wetten aufstellen. Heute war so ein merkwürdiger Teenie mit feurig roten Haaren da, der sich als \ron ausgab und behauptete, er könne Personen verschwinden lassen. Ich wollte gerade von der Bühne gehen, um ihm die Show zu überlassen, da wurde mir plötzlich ganz schummrig und ich hatte das Gefühl, zu schweben. Plötzlich wurde alles schwarz, und... als ich die Augen wieder aufmachte, war ich nicht mehr auf meiner Bühne, sondern inmitten einem Haufen komischer Gestalten. Ja, und den Rest kennt ihr ja \dots}
\end{drama}

\StageDir{Nachdem \thomas geendet hat, sehen sich \schneewittchenh, \remyh und \doch eine Weile fragend an. Erneut ergreift \schneewittchenh das Wort.}

\begin{drama}
\schneewittchenx{Also wenn ihr mich fragt \dots\ So merkwürdig kann kein Spion sein.}
\remyx{Ich bin mir nicht sicher, aber ich denke, wir sollten die anderen fragen. Ihn hierzulassen wäre herzlos.}
\schneewittchenx{Herzlos wäre, nichts an seinem Outfit zu verändern!}
\docx{Ich will dich ja nicht unterbrechen, mein Schatz, aber ich denke, es ist zu gefährlich, weiter hier herumzustehen...}
\remyx{\doc hat Recht, wir sollten uns zum Geheimversteck aufmachen.}
\schneewittchenx{In Ordnung. \direct{Zu \thomas} Aber he! Wie heißt du eigentlich?}
\thomasx{\thomas ist mein Name. Und du bist wirklich \schneewittchen?}
\schneewittchenx{Ja, natürlich! Oder hast du sonst schonmal jemanden wie mich getroffen?}
\remyx{\direct{Unterbricht die beiden} Los jetzt, wir müssen gehen!}
\end{drama}

\StageDir{Die vier verlassen die Bühne. Licht aus.}