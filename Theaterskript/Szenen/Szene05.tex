% This file is part of TenSing-Moers/Theaterskript2017.
%
% TenSing-Moers/Theaterskript2017 is free content: you can redistribute and/or
% modify it under the terms of the cc-by-nc-sa (Creative Commons
% Attribution-NonCommercial-ShareAlike) as released by the
% Creative Commons organisation, version 4.0.
%
% TenSing-Moers/Theaterskript2017 is distributed in the hope that it will be useful,
% but without any warranty.
%
% You should have received a copy of the cc-by-nc-sa-license along
% with this copy of TenSing-Moers/Theaterkskript2017. If not, see
% <https://creativecommons.org/licenses/by-nc-sa/4.0/legalcode>.
%
% Copyright TenSing Moers and all whose work and <3 went in this project.
\renewcommand{\sechl}{\tikz[overlay]{ \draw[fill=Orange1] (-.6\textwidth,-1.2em) rectangle (1.15\textwidth,1em);}}%\draw (-.75\textwidth,-1ex) -- (\textwidth,-1ex); \draw (-\textwidth,1em) -- (\textwidth,1em)}%}
\Scene{Pilze sammeln mit ungeahnten Folgen}
\renewcommand{\sechl}{\tikz[overlay]{ \draw[fill=Orange1] (-.6\textwidth,-1ex) rectangle (1.15\textwidth,1em);}}%\draw (-.75\textwidth,-1ex) -- (\textwidth,-1ex); \draw (-\textwidth,1em) -- (\textwidth,1em)}%}

\DisplayPersons
%\chars{Helge Schneider}
\ort{Im Wald}
\requ{Blumen, Plüschkaninchen für Pocahontas, leeres Honigglas (für Winnie Pooh). Busch, Fesseln, Pilze, Weidenkorb für die Pilze}

\StageDir{\meridah, \olafh und \doch laufen über die Bühne zu \pocahontash. \olafh findet alles toll, bleibt ständig stehen und freut sich über irgendetwas.}
\mikroan{1,3,4,5,}

\begin{drama}
\meridax{Meine Güte, \olaf, komm jetzt endlich!}
\olafx{Aber guck doch mal, diese tolle Blume!}
\meridax{\direct{seufzt} Hier sind viele tolle Blumen, \olaf. Wir kommen nie voran, wenn du wegen jeder Kleinigkeit stehen bleibst! \direct{entdeckt \pocahontas} Hallo \pocahontas, wir sollen Pilze sammeln, hilfst du uns?}
\pocahontasx{\direct{steht auf, nachdem sie einem Kaninchen noch ein letztes Mal über's Fell gestrichen hat} Dir wird es bald besser gehen \dots\ \direct{wendet sich zu den anderen} Pilze sagt ihr? Jaa, was denn für welche? Ihr wisst ja, dass ihr nur von ganz bestimmten Pilzsorten welche mitnehmen dürft, sonst zerstört ihr das Gleichgewicht! Und bedankt euch immer bei den Pilzen!}
\meridax{Ja ja, wissen wir doch, das hast du uns schon tausendmal gesagt! \doc, was für Pilze sollen wir nochmal sammeln?}
\docx{Pfifferlinge und Steinpilze, glaube ich.}
\meridax{\direct{zeigt auf eine Stelle am Boden} Oh, seht mal, da sind schon welche! Kommt, wir nehmen die mit! \direct{will sich hinknien}}
\pocahontasx{Halt, stopp! Was habe ich euch denn gerade gesagt? \direct{springt zu \merida} Ihr dürft das Gleichgewicht nicht stören! Seht ihr nicht, dass diese Steinpilze gerade zu Mittag essen?}
\customx{\meridah und \doch}{Ähh\dots}
\pocahontasx{War ja klar. Ihr seid solche Rüpel!}
\meridax{Aber\dots}
\pocahontasx{Kein Aber. Ich sammle die Pilze wohl besser selbst, ihr versteht das doch eh nicht. Wir sehen uns nachher im Versteck! \direct{geht ab. Man hört sie noch rufen:} Und passt auf das Kaninchen auf, hört ihr! Irgendwas hat den armen kleinen verletzt!}
\meridax{\direct{ahmt \pocahontas Worte nach} Oh, sie essen gerade zu Mittag!}
\end{drama}

\StageDir{\meridah und \doch lachen}

\begin{drama}
\meridax{Komm, wir nehmen trotzdem welche mit. So gute Steinpilze kann doch keiner stehen lassen\dots}
\end{drama}

\StageDir{\meridah und \doch knien sich hin und sammeln die Pilze}

\begin{drama}
\docx{\direct{genervt} Wo ist denn Olaf schon wieder hin?}
\meridax{Keine Ahnung\dots\ Er war doch gerade noch bei uns?}
\docx{Bestimmt ist er schon wieder irgendeinem Schmetterling nachgelaufen\dots\ Wir gehen ihn wohl besser suchen.}
\end{drama}

\StageDir{Im Hintergrund taucht \olafh auf, der auf \winnieh zuläuft, \winnieh schaut in ein leeres Honigglas. \doch zieht \meridah hinter einen Busch.}

\begin{drama}
\meridax{Da vorne ist er ja! Aber\dots\ Wer ist denn der andere?}
\docx{Das ist \winnie, ich habe gehört, er steht auf der Seite von \mickey\dots}
\meridax{Wirklich? Aber\dots\ Was macht er hier? Und sollten wir nicht eingreifen, bevor \olaf etwas passiert?}
\docx{Ich stimme dir zu, wir sollten nicht tatenlos zusehen. Aber was machen wir mit \winnie?}
\end{drama}

\StageDir{\olafh bleibt stehen und betrachtet eine Blume.}

\begin{drama}
\meridax{Wir sollten ihn überfallen und ins Versteck bringen, vielleicht bekommen wir etwas aus ihm heraus!}
\docx{Oder wir überfallen ihn einfach von hinten, wenn er von \olaf abgelenkt ist.}
\meridax{Na gut, das ist irgendwie einfacher. Machen wir es so. Aber schnell, bevor \olaf etwas passiert!}
\end{drama}

\StageDir{\doch und \meridah schleichen sich von hinten an \winnieh und \olafh heran.}

\begin{drama}
\olafx{\direct{bemerkt \winnie, sehr aufgeregt} Ohh, wer bist du denn?!}
\end{drama}

\StageDir{\winnieh sieht auf und sieht Olaf. In diesem Moment stürzen sich \meridah und \doch von hinten auf ihn und ringen ihn zu Boden.}

\begin{drama}
\olafx{\direct{entsetzt} Hey! Was macht ihr denn da? Ihr tut ihm noch weh!}
\meridax{Alles klar, er ist sicher verschnürt!}
\end{drama}

\StageDir{\winnieh wehrt sich heftig.}

\begin{drama}
\meridax{Halt still! \direct{zu \doc} Wir sollten ihn schleunigst zu den anderen bringen.}
\olafx{Aber\dots\ Warum tut ihr sowas?}
\docx{Olaf, du findest immer alles und jeden toll, aber der hier ist einer von \mickey's Schergen. Der ist gefährlich!}
\olafx{\direct{traurig} Aber der sah doch überhaupt gar nicht gefährlich aus!}
\meridax{Ist mir egal, ob er gefährlich aussieht, oder nicht, wir sollten hier weg!}
\docx{\olaf, nimm du die Pilze, wir gehen zurück!}
\end{drama}

\mikroaus{1,3,4,5}
\StageDir{\olafh nimmt den Korb mit den Pilzen und trottet hinter den anderen her von der Bühne. Licht aus.}