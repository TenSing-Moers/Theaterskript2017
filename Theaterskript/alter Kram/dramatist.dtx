%\iffalse
%% dramatist.dtx
%% Copyright (C) 2003-2014 Massimiliano Dominici
%
% This program is free software; you can redistribute it and/or modify
% it under the terms of the GNU General Public License as published by
% the Free Software Foundation; either version 2 of the License, or
% (at your option) any later version.
%
% This program is distributed in the hope that it will be useful,
% but WITHOUT ANY WARRANTY; without even the implied warranty of
% MERCHANTABILITY or FITNESS FOR A PARTICULAR PURPOSE. See the
% GNU General Public License for more details.
%
% You should have received a copy of the GNU General Public License
% along with this program; if not, write to the Free Software
% Foundation, Inc., 675 Mass Ave, Cambridge, MA 02139, USA.
%
%    Copyright (C) 2003-2004 Massimiliano Dominici.
%    Permission is granted to copy, distribute and/or modify this document
%    under the terms of the GNU Free Documentation License, Version 1.2
%    or any later version published by the Free Software Foundation;
%    with no Invariant Sections, no Front-Cover Texts, and no Back-Cover Texts.
%    A copy of the license is included in the section entitled ``GNU
%    Free Documentation License''.
%
% This program consists of the files dramatist.dtx and dramatist.ins
%
%<*driver>
\documentclass[a4paper]{ltxdoc}
\usepackage[T1]{fontenc}
\usepackage[latin1]{inputenc}
\usepackage{ae}
\usepackage{dramatist}
\GetFileInfo{dramatist.sty}
\makeatletter
\c@IndexColumns=2
\def\DescribeOpt{\leavevmode\@bsphack\begingroup\MakePrivateLetters
    \Describe@Opt}
\def\Describe@Opt#1{\endgroup
              \marginpar{\raggedleft\PrintDescribeOpt{#1}}%
              \SpecialOptIndex{#1}\@esphack\ignorespaces}
\def\SpecialOptIndex#1{\@bsphack
    \index{#1\actualchar{\protect\ttfamily#1}
           (option)\encapchar usage}%
    \index{options:\levelchar{\protect\ttfamily#1}\encapchar
           usage}\@esphack}
\newcommand{\PrintDescribeOpt}{\PrintDescribeEnv}
\makeatother
\OnlyDescription
\EnableCrossrefs
\CodelineIndex
\RecordChanges
\providecommand\envname[1]{\textsf{#1}}
\providecommand\clsname[1]{\textsf{#1}}
\providecommand\pkgname[1]{\textsf{#1}}
\providecommand\Filename[1]{\textsf{#1}}
%\renewcommand{\theenumi}{\alph{enumi}}
\begin{document}
    \DocInput{dramatist.dtx}
\end{document}
%</driver>
%\fi
% \CheckSum{523}
%% \CharacterTable%%  {Upper-case    \A\B\C\D\E\F\G\H\I\J\K\L\M\N\O\P\Q\R\S\T\U\V\W\X\Y\Z
%%   Lower-case    \a\b\c\d\e\f\g\h\i\j\k\l\m\n\o\p\q\r\s\t\u\v\w\x\y\z
%%   Digits        \0\1\2\3\4\5\6\7\8\9
%%   Exclamation   \!     Double quote  \"     Hash (number) \#
%%   Dollar        \$     Percent       \%     Ampersand     \&
%%   Acute accent  \'     Left paren    \(     Right paren   \)
%%   Asterisk      \*     Plus          \+     Comma         \,
%%   Minus         \-     Point         \.     Solidus       \/
%%   Colon         \:     Semicolon     \;     Less than     \<
%%   Equals        \=     Greater than  \>     Question mark \?
%%   Commercial at \@     Left bracket  \[     Backslash     \\
%%   Right bracket \]     Circumflex    \^     Underscore    \_
%%   Grave accent  \`     Left brace    \{     Vertical bar  \|
%%   Right brace   \}     Tilde         \~}
%%
%
%\changes{v1.0}{2003/08/03}{First public release.}
%\changes{v1.1}{2004/01/19}{Added support for line numbering; added
%a \cs{speaker} command; changes made to the \envname{drama*}
%environment; made \cs{act}, \cs{scene} and \cs{DramPer} more compliant
%to the standard document division commands.}
%\changes{v1.2}{2004/05/10}{Added support for \pkgname{poemscol}
%package; introduced an \emph{uppercase} series of sectioning
%commands; first argument in \cs{Character} made optional; added
%environment (\envname{CharacterGroup}) for characters groups in
%the \emph{Dramatis Person\ae} list; added various hooks for user
%customization.}
%\changes{v1.2a}{2005/05/21}{Fixed a bug in the vertical spacing of
%\cs{DramPer}. Changed name of counter \textsf{linenumber} in
%\textsf{verselinenumber} in order to mantain compatibility with
%package \pkgname{poemscol}}
%\changes{v1.2c}{2005/07/26}{Fixed two bugs: now the package works
%correctly with the spanish extension of \pkgname{babel} and a
%\cs{\meta{name}} command at the end of a \cs{direct} macro no more gives
%an unwanted space. Fixed a typo in the author e-mail address.}
%\changes{v1.2d}{2005/12/01}{Fixed three bugs: restored a missing
%backslash in \cs{dirdelimiter}; customized lengths in character groups
%inside a Dramatis Personae list now work correctly; \cs{speaksdel} is
%now appended to characters' label in verse drama environment too.}
%
%\DoNotIndex{\@afterheading,\@afterindentfalse,\@break@tfor,\@empty}
%\DoNotIndex{\@ifstar,\@let@token,\@namedef,\@nameuse,\@nil}
%\DoNotIndex{\@tempswafalse,\@tempswatrue,\@tfor,\\}
%\DoNotIndex{\ae,\baselineskip,\begin,\bgroup,\centering}
%\DoNotIndex{\csname,\def,\do,\egroup,\else,\emph,\end}
%\DoNotIndex{\expandafter,\fi,\futurelet,\hspace,\if@tempswa}
%\DoNotIndex{\ifnum,\ifvmode,\ifx,\InputIfFileExists,\item}
%\DoNotIndex{\itemindent,\itemsep,\labelsep,\labelwidth}
%\DoNotIndex{\Large,\large,\leftmargin,\let,\makelabel}
%\DoNotIndex{\NeedsTeXFormat,\newcommand,\newcounter,\newenvironment}
%\DoNotIndex{\newif,\newlength,\newpage,\nobreak,\normalfont,\normallineskip}
%\DoNotIndex{\PackageError,\PackageWarningNoLine,\par,\parbox,\parindent}
%\DoNotIndex{\ProvidesPackage,\refstepcounter,\relax,\renewcommand}
%\DoNotIndex{\reserved@a,\reserved@b,\reserved@c,\roman,\scshape,\setlength}
%\DoNotIndex{\space,\stepcounter,\textwidth,\unskip,\value,\vskip,\vspace,\z@}
%\DoNotIndex{\@centercr,\@ifundefined,\addcontentsline,\addvspace,\DeclareOption}
%\DoNotIndex{\endcsname,\if@openright,\secdef,\setcounter,\providecommand,\ProcessOptions}
%\DoNotIndex{\parskip,\PackageWarning,\@ifpackageloaded,\clearpage,\cleardoublepage,\em}
%\DoNotIndex{\global,\ht,\left,\leftmargini,\newsavebox,\raggedright,\right,\Roman,\rule}
%\DoNotIndex{\tbox,\usebox,\xdef,\@undefined,\c@secnumdepth,\endlist,\list,\m@ne,\newdimen}
%\DoNotIndex{\parsep,\thispagestyle}
%\expandafter\DoNotIndex\expandafter{\string\}}
% \begingroup
%     \makeatletter
%     \lccode`9=32\relax
%     \lowercase{%^^A
%       \edef\x{\noexpand\DoNotIndex{\@backslashchar9}}%^^A
%     }%^^A
%   \expandafter\endgroup\x%
%\title{The \Filename{dramatist} package\thanks{This file has version
%number \fileversion{} dated \filedate{}.} \\User Guide}
%\author{Massimiliano Dominici\\\texttt{mlgdominici@gmail.com}}
%\date{\filedate}
%\maketitle
%
%\begin{abstract}
%The present package provides support for drama both in verse and
%in prose. The following facilities are given: two environments for
%typesetting dialogues in prose or in verse; new document divisions
%corresponding to acts and scenes; macros that control the
%appearance of characters and stage directions; and automatical
%generation of a \emph{dramatis person\ae} list.
%\end{abstract}
%
%\tableofcontents
%
%\section{Introduction}
%
%The edition of a drama requires special treatment for many
%typographical elements. The purpose of the present package is that
%of providing full support for these specialities. So, besides the
%standard document divisions, new ones are introduced reflecting
%the peculiar nature of the document itself; environments are
%provided for introducing dialogues, and a set of macros is placed
%at the user's disposal to handle characters, automatically
%generate a \emph{dramatis person\ae} list, and control the
%appearance of stage directions. All these features I have tried to
%make fully customizable, with the idea that typographical
%conventions are hints rather than laws, and the fully conscious
%user should be enabled to override them.
%
%The decision to write a package rather than a class is due to
%similar considerations about user's freedom. The package strictly
%provides what is meant in his name and doesn't involve itself in
%the layout design of the document. This task is left to the class
%chosen by the user. In particular, the package does not provide
%explicit support for text in verse, though it provides support for
%those features that are peculiar to a \emph{drama} in verse.
%However, \pkgname{dramatist} is integrated with the main packages
%dealing with verse (such as \pkgname{verse} or
%\pkgname{poemscol}), so that the user can, for instance, use line
%numbering defined by one of the aforesaid packages in a meaningful
%way inside a \envname{drama*} environment.
%
%\section{User interface}
%
%\subsection{Package Options}
%\DescribeOpt{lnpa}\DescribeOpt{lnps} The package provides two
%options, both concerning line numbering in verse drama. By default
%none of the options is used and the counter holding the line
%number is not reset throughout the document. If you like it better
%you can choose line numbering per act or per scene issuing one of
%the options, namely: |lnpa| or |lnps|. Issuing the options when
%typesetting a play in prose, has no effect on the document, but a
%package warning is typed in the |log| file every time a |drama|
%environment is called.
%
%\subsection{The \envname{drama} environment}
%
%The |drama| environment is the heart of the package. Two versions,
%of this environment, are provided: the normal version, used for
%typesetting dramas in prose, and the starred version
%(\envname{drama*}) for typesetting dramas in verse.
%\DescribeEnv{drama} The unstarred form arranges the items given by
%the macros for defining characters (see Section~\ref{sec:cdc}) in
%a sort of description-like environment -- but the parameters can
%be managed and adjusted to get every kind of list the user
%desires. These are the hooks provided for customizing the look of
%the environment\footnote{For this parametrization of the
%\envname{drama} environment I'm in debt with Christian Ebert.}:
%\begin{description}
%    \item[\cs{speakswidth}] is the width of the label in which the
%    name of the character is printed;
%    \item[\cs{speaksindent}] is the indentation of that label;
%    \item[\cs{Dlabelsep}] is the space between the label and the
%    text;
%    \item[\cs{Dparsep}] is the space between paragraphs inside the
%    dialogue;
%    \item[\cs{speechskip}] is the space between two subsequent
%    speeches.
%\end{description}
%You can see default settings for these macros in
%Table~\ref{tab:drama}. A conditional expression checks the
%eventual presence of a line numbering option, in which case a
%warning is sent to the user in the |log| file about the
%meaninglessness of the option.
%
%\begin{table}
%\begin{center}
%\DeleteShortVerb{\|}
%\begin{tabular}{|c|c|}
%\hline
% \textbf{Commands}        &        \textbf{Default settings}               \\
%\hline
%\hline
% \verb+\speakswidth+      &              \verb+\z@+                        \\
%\hline
% \verb+\speaksindent+     &         \verb+-\leftmargin+                    \\
%\hline
% \verb+\speechskip+       &            \verb+\itemsep+                     \\
%\hline
% \verb+\Dparsep+          &              \verb+\z@+                        \\
%\hline
% \verb+\Dlabelsep+        &           \verb+\labelsep+                     \\
%\hline
%\end{tabular}
%\caption{Sectioning commands}\label{tab:drama}
%\end{center}
%\end{table}
%\MakeShortVerb{\|}
%
%\DescribeEnv{drama*} The starred form switches to |\@drversetrue|
%(this is used by the commands that define characters -- see
%Section~\ref{sec:cdc}), calls the \envname{verse} environment --
%or the \envname{poem} environment if \pkgname{poemscol} package
%has been loaded: this is automatically recognized by the package
%and needs no additional option --, and, if |\poemlines| is
%defined\footnote{This is provided by the \clsname{memoir}
%\cite{mem} class and the \pkgname{verse} \cite{ver} package for
%line numbering purpose.}, the value of the |poemline| counter is
%restored at the beginning and saved at the end of the environment
%(this is needed because, by default, the aforesaid counter is
%reset to 1 every time the \envname{verse} environment is called).
%In case \pkgname{poemscol} has been loaded, the same operations
%are performed for counters \textsf{linenumber} and
%\textsf{printlineindex}.
%
%\subsection{Sectioning Commands}
%
%The package provides two series of commands in order to get a
%proper sectioning of the text: a \emph{lowercase} series and an
%\emph{uppercase} series. The difference between them is that the
%\emph{lowercase} form takes no mandatory argument and can be used
%absolutely, while the \emph{uppercase} form take one mandatory
%argument and should be used only when a title is specified as a
%part of the act/scene heading. I chose to introduce this peculiar
%form of sectioning commands without argument (and as the default
%one) because I think that in most cases the user only wants to get
%something like \emph{Act I} and should not bore himself issuing a
%pair of curly braces.
%
%So \DescribeMacro{\act}|\act| and \DescribeMacro{\scene}|\scene|
%print by default only the act or scene name (e.g.: Act) and its
%ordinal number. They (and |\DramPer| also) may take an optional
%argument: this feature is useful for inserting footnotes or
%endnotes in the act/scene headings, but causes an undesirable
%consequence: the user should issue a blank line after each of
%these commands when used without argument (two blank lines for
%|\DramPer| followed by a void |\scene|).
%
%\DescribeMacro{\Act}|\Act| and \DescribeMacro{\Scene}|\Scene| take
%as mandatory argument the title of the act or of the scene. An
%optional argument has the same meaning as for standard sectioning
%commands (|\chapter|, |\section|, etc.). The title is printed by means
%of |\printacttitle| or |\printscenetitle|.
%
%Each command is fully customizable, in the style of Peter Wilson's
%|memoir| class (see for the documentation \cite{mem}), via
%commands like \DescribeMacro{\printactname}|\printactname| or
%\DescribeMacro{\printactnum}|\printactnum|. These commands can be
%redefined by the user according with his own desire. Table
%\ref{tab:com} shows the commands and their default settings.
%
%\begin{table}
%\begin{center}
%\DeleteShortVerb{\|}
%\begin{tabular}{|c|c|}
%\hline
% \textbf{Commands}        &        \textbf{Default settings}               \\
%\hline
%\hline
%   \verb+\printactname+   &     \verb+\centering\actnamefont \actname+     \\
%\hline
%   \verb+\printactnum+    &           \verb+\actnumfont \theact+           \\
%\hline
%   \verb+\printacttitle+  &           \verb+\acttitlefont #1+             \\
%\hline
%     \verb+\actname+      &                    Act                         \\
%\hline
%   \verb+\actnamefont+    &             \verb+\scshape\Large+              \\
%\hline
%    \verb+\actnumfont+    &              \verb+\actnamefont+               \\
%\hline
%    \verb+\acttitlefont+  &              \verb+\actnamefont+               \\
%\hline
%      \verb+\theact+      &               \verb+\roman{act}+               \\
%\hline
%  \verb+\actcontentsline+ &           \verb+\actname\ \theact+             \\
%\hline
%  \verb+\printscenename+  &   \verb+\centering\scenenamefont \scenename+   \\
%\hline
%  \verb+\printscenenum+   & \verb+\scenenumfont \theact\intersep\thescene+ \\
%\hline
%  \verb+\printscenetitle+ &           \verb+\scenetitlefont #1+            \\
%\hline
%  \verb+\scenenamefont+   &             \verb+\scshape\large+              \\
%\hline
%   \verb+\scenenumfont+   &             \verb+\scenenamefont+              \\
%\hline
%  \verb+\scenetitlefont+  &             \verb+\scenenamefont+              \\
%\hline
%    \verb+\scenename+     &                   Scene                        \\
%\hline
%     \verb+\thescene+     &              \verb+\roman{scene}+              \\
%\hline
%\verb+\scenecontentsline+ &         \verb+\scenename\ \thescene+           \\
%\hline
%     \verb+\printsep+     &                   \verb*+\ +                   \\
%\hline
%     \verb+\intersep+     &                 \verb*+\ --\ +                 \\
%\hline
%\end{tabular}
%\caption{Sectioning commands}\label{tab:com}
%\end{center}
%\end{table}
%\MakeShortVerb{\|}
%
%\DescribeMacro{\actmark}|\actmark| and \DescribeMacro{\scenemark}
%|\scenemark|, defined by default to do nothing, can be useful for
%printing marks in the headers, and have the same meaning of
%|\chaptermark| and |\sectionmark| in the standard classes.
%
%A starred version also is provided for |\Act|, |\act|, |\Scene|
%and |\scene|. As in standard classes the starred form does not
%make an entry for the table of contents, and does not print the
%section mark in the headers.
%
%\subsection{Defining characters\label{sec:cdc}}
%
%The introduction of a new character is made by the command
%\DescribeMacro{\Character} |\Character|. It takes three arguments:
%the first, optional\footnote{The argument, mandatory until version
%1.1, has been made optional by suggestion of Christian Ebert in
%version 1.2. This avoids issuing a \cs{Character} command with a
%dummy first argument \emph{after} \cs{DramPer} when you need a
%character not appearing in the \emph{Dramatis Person\ae} list.},
%is the entry for the list of \emph{Dramatis Person\ae}, the second
%is the name appearing in the text and the third is the base for
%the construction of the commands typesetting the occurrence of
%that name in the stage direction and as a speaker. Shortly, if
%\meta{name} is given as third argument, the macro will return the
%following commands: \verb|\|\meta{name} is used in stage
%direction, \verb|\|\meta{name}|speaks| is used as speaker.
%
%The first argument, when present, is passed to the macro
%\DescribeMacro{\DramPer}|\DramPer|, printing the list of
%\emph{Dramatis Person\ae}. The parameters of this macro are also
%customizable; the list of default settings can be seen in
%Table~\ref{tab:char}. Other parameters related to the
%customization of \verb|\|\meta{name} and
%\verb|\|\meta{name}|speaks| commands are added.
%
%You can also use a \DescribeMacro{\speaker}
%|\speaker{|\meta{name}|}| command in the place you want the
%character with name \meta{name} to appear (in this case no command
%is defined to print the name of the character inside a stage
%direction).
%
%\subsubsection{Grouped characters}
%
%Characters, in the \emph{Dramatis Person\ae} list, may need to be
%grouped under a common denomination\footnote{Thanks to Christian
%Ebert for having submitted the problem to my attention.}. For this
%occurrence the package provides an environment,
%\DescribeMacro{CharacterGroup} \envname{CharacterGroup}, taking,
%as mandatory argument the denomination common to each character
%belonging to the current group. Inside this environment the
%characters have to be defined by \DescribeMacro{\GCharacter}
%|\GCharacter| whose syntax is the same of |\Character|, except
%that the first argument is here, obviously, mandatory. The result
%will be that the characters will be grouped by a big parentheses
%on whose right will be printed, centered, the common denomination.
%The user can define the amount of space reserved to the characters
%names, the parentheses and the common denomination by means of
%\DescribeMacro{\CharWidth} |\CharWidth|,
%\DescribeMacro{\ParenWidth} |\ParenWidth| and
%\DescribeMacro{\GroupWidth} |\GroupWidth|.
%
%\bigskip
%
%The commands appearing in this section (especially |\DramPer| and
%|\Character|, i.e. the most crucial part of the whole work) have
%been inspired by Matt Swift's package |drama|.
%
%\begin{table}
%\begin{center}
%\DeleteShortVerb{\|}
%\begin{tabular}{|c|c|}
%\hline
% \textbf{Commands}      & \textbf{Default settings}                      \\
%\hline
%\hline
% \verb+\printcasttitle+ & \verb+\centering\casttitlefont \casttitlename+ \\
%\hline
% \verb+\casttitlefont+  &             \verb+\Large\scshape+              \\
%\hline
% \verb+\casttitlename+  &                 Dramatis Person\ae             \\
%\hline
%    \verb+\castfont+    &               \verb+\normalfont+               \\
%\hline
%    \verb+\namefont+    &                \verb+\scshape+                 \\
%\hline
%   \verb+\speaksfont+   &                \verb+\scshape+                 \\
%\hline
%    \verb+\speaksdel+   &                                                \\
%\hline
%\end{tabular}
%\caption{Parameters for characters commands}\label{tab:char}
%\end{center}
%\end{table}
%\MakeShortVerb{\|}
%
%\subsection{Stage direction}
%
%Two commands are provided for printing stage directions: the
%first, \DescribeMacro{\StageDir}|\StageDir| is used for the very
%setting of the stage and calls a |quote| environment. The second,
%\DescribeMacro{\direct} is used for specifications in the middle
%of the speaker's text. In connection with the |verse| environment
%a starred version exists also to be used at the end of a
%stanza\DescribeMacro{\direct*}\footnote{This works only with the
%|verse| environment provided by the |verse| package and by the
%|memoir| class. You can use the normal, not starred version of the
%command with the standard |verse| environment}. A
%\DescribeEnv{stagedir} |stagedir| environment is finally provided
%for extremely long stage directions: it differs in nothing from
%his command version.
%
%|\StageDir| and the |stagedir| environment can be customized by
%means of \DescribeMacro{\StageDirConf} |\StageDirConf|. The
%command takes two arguments related to the code to be executed at
%the beginning and at the end of |\StageDir| or |stagedir|.
%
%\subsection{Local configuration file}
%
%As my chief aim has been the one of giving the user the support
%for a complete customization of the provided commands, I have
%taken under consideration the case of a stable local configuration
%set up by the user. So I introduced the possibility of reading a
%set of user definitions from a local configuration file called
%\marginpar{\raggedleft\small |dramatist.cfg|}|dramatist.cfg|. You
%must create this file, if you need it, and place it in a suitable
%directory (either the working directory or the package directory);
%if you don't need it, no problem: the package will merely print in
%your log file a message claiming the absence of such a file, but
%nor interruptions neither error will take place.
%
%\subsection{Acknowledgements}
%
%I must acknowledge a debt of inspiration toward both Peter
%Wilson's all purpose class package |memoir| and Matt Swift's
%|drama|. While the former inspired me with a peculiar care toward
%complete customization, the latter was the formal model and the
%source of solution for many among the problems which arose in the
%proceeding of the code writing. I highly recommend the use of the
%class |memoir| and the study of Matt Swift's ambitious bundle
%|Frankenstein| (but I hope you want use |dramatist| package instead!).
%
%I also wish to thank Christian Ebert, whose suggestions have been
%so useful to me in solving -- and often in merely recognizing --
%many problems, and whose help and stimulating conversation is at
%the basis of the present (v1.2) revision of the package.
%
%\subsection{Known bugs}
%
%The user should issue an empty line after |\act| or |\scene| and
%\emph{two} empty lines after |\DramPer| (especially when it's
%followed by |\scene| used without argument).
%
%Using option |lnpa| or |lnps| together whith package
%\pkgname{hyperref} causes a long series of warning to be typed in
%the |log| file. This is due, I think, because \pkgname{hyperref}
%finds duplicates of the same entry every time the \envname{drama*}
%environment resets the |poemline| counter. However, there are no
%effects on the document, because the counter itself is by no way
%used by \pkgname{hyperref}.
%
%If you encounter new bugs, or have suggestions about the solution
%of the known ones, please send me a mail to this address:
%\texttt{mlgdominici@gmail.com.}
%
%\section{Code}
%
%\subsection{Package identification}
%
%    \begin{macrocode}
%<*dramatist>
\ProvidesPackage{dramatist}[2014/12/18 v1.2e Package for typesetting drama -- Author: Massimiliano Dominici]
\NeedsTeXFormat{LaTeX2e}
\RequirePackage{xspace}
%    \end{macrocode}
%\subsection{Conditionals, options and counters}
%The following line checks wether a class defining |\if@openright|
%has been loaded; if not, it defines that conditional expression.
%    \begin{macrocode}
\@ifundefined{if@openright}{\newif\if@openright}{}
\newif\if@drverse
%    \end{macrocode}
%The following lines provide support for the \pkgname{poemscol}
%package.
%    \begin{macrocode}
\newif\if@poemscol
\@ifpackageloaded{poemscol}{\@poemscoltrue}{\@poemscolfalse}
%    \end{macrocode}
%The conditional |\if@stagedir| is switched to true at the end of a
%stage direction.
%    \begin{macrocode}
\newif\if@stagedir
%    \end{macrocode}
%The two options for line numbering are defined to switch to true a
%conditional expression.
%    \begin{macrocode}
\newif\if@lnpa
\newif\if@lnps
\newif\if@lnpd
\DeclareOption{lnpa}{\@lnpatrue}
\DeclareOption{lnps}{\@lnpstrue}
\ProcessOptions
%    \end{macrocode}
%This is needed  for saving and restoring the value of the
%|poemline| counter in the case it is defined and used.
%    \begin{macrocode}
\newcounter{storelineno}
\setcounter{storelineno}{0}
\if@poemscol\else
\refstepcounter{storelineno}\fi
\newcounter{storeprintlineindex}
%    \end{macrocode}
%    \begin{macrocode}
\newcounter{character}
\newcounter{temp}
\newcounter{gtemp}
\newcounter{act}
\newcounter{scene}[act]
\renewcommand{\theact}{\roman{act}}
\renewcommand{\thescene}{\roman{scene}}
%    \end{macrocode}
%\subsection{Environments}
%\begin{macro}{drama}
%\changes{v1.2}{2004/05/10}{Added hooks for customization.}
%The unstarred version of the |drama| environment defines a list
%with negative item indentation and whose label is the speaker's
%name. A previous check is made for an option and, in the case it
%has been issued, a warning is typed out to the |log| file. Hooks
%for user customization are provided: \DescribeMacro{\speakswidth}
%|\speakswidth| is the width of a label in which the name of the
%character is printed; \DescribeMacro{\speaksindent}|\speaksindent|
%is the indentation of the same label;
%\DescribeMacro{\Dlabelsep}|\Dlabelsep| is the space between this
%label and the text of the dialogue;
%\DescribeMacro{\Dparsep}|\Dparsep| controls the space between
%paragraphs inside the
%dialogue;\DescribeMacro{\speechskip}|\speechskip|  controls the
%space between two subsequent
%speeches.\DescribeMacro{\speakslabel}|\speakslabel| formats the
%appearance of the name of the character.
%    \begin{macrocode}
\newenvironment{drama}{%
    \if@lnpa
    \PackageWarning{dramatist}{\lnpwarning{a}}
    \fi
    \if@lnps
    \PackageWarning{dramatist}{\lnpwarning{s}}
    \fi
    \list{}{%
        \labelwidth\speakswidth
        \itemindent\speaksindent
        \itemsep\speechskip
        \parsep\Dparsep
        \labelsep\Dlabelsep
        \let\makelabel\speakslabel}
    } {\endlist}
%    \end{macrocode}
%\end{macro}
%\begin{macro}{drama*}
%\changes{v1.1}{2004/01/19}{Environment \envname{drama*} has been
%completely restyled. Now, it automatically calls environment
%\envname{verse}.}
%\changes{v1.2}{2004/05/10}{Added support for \pkgname{poemscol}.}
%The starred version calls the \envname{verse} environment (or the
%\envname{poem} environment if \pkgname{poemscol} is loaded), after
%switching to |\@drversetrue|, controls line numbering, if any,
%and, after closing \envname{verse}, restores |\@drversefalse|.
%    \begin{macrocode}
\@namedef{drama*}{%
    \@drversetrue
    \if@poemscol
        \begin{poem}
        \setcounter{verselinenumber}{\value{storelineno}}
        \setcounter{printlineindex}{\value{storeprintlineindex}}
    \else
        \begin{verse}
    \fi
    \ifx\poemlines\@undefined\else
        \setcounter{poemline}{\value{storelineno}}
    \fi}
\@namedef{enddrama*}{%
    \ifx\poemlines\@undefined\else
        \setcounter{storelineno}{\value{poemline}}
    \fi
    \if@poemscol
        \end{poem}
        \setcounter{storelineno}{\value{verselinenumber}}
        \setcounter{storeprintlineindex}{\value{printlineindex}}
    \else
        \end{verse}
    \fi
    \@drversefalse}
%    \end{macrocode}
%\end{macro}
%\subsection{Sectioning commands}
%
%The sectioning commands |\act| and |\scene| have been made wholly
%customizable via |\m@ke@cthead| and |\m@kescenehead| just like the
%sectioning commands of |memoir| class (see \cite{mem} for further
%details).
%
%|\phantomsection| is needed for compatibility with the
%\envname{hyperref} package. It is defined to do nothing when
%\envname{hyperref} is not loaded.
%    \begin{macrocode}
\providecommand\phantomsection{}
\newcommand\actmark[1]{}
\newcommand\scenemark[1]{}
\newcommand\drampermark[1]{}
%    \end{macrocode}
%\begin{macro}{\@openact}
%|\@openact| must check if a class defining |\if@openright| has
%been loaded. In this case it provides an if statement to control
%switching between \textsf{openany} and \textsf{openright}
%behaviour. By default, the option loaded with the class is
%inherited. If the class loaded behaves like \clsname{article} only
%the \textsf{openany} option is allowed. According to the option
%loaded for line numbering, |\@openact| performs the needed
%operations.
%    \begin{macrocode}
\newcommand\@openact{%
    \@ifundefined{if@openright}{\clearpage}{%
        \if@openright
            \clearpage{\thispagestyle{empty}\cleardoublepage}
        \else
            \clearpage
        \fi}
    \thispagestyle{plain}
    \refstepcounter{act}
    \if@lnpa
        \setcounter{storelineno}{0}
        \if@poemscol
            \setcounter{storeprintlineindex}{0}
        \else
            \refstepcounter{storelineno}
        \fi
    \fi
}
%    \end{macrocode}
%\end{macro}
%\begin{macro}{\act}
%\changes{v1.1}{2004/01/19}{\cs{act} is now defined in a standard way
%and has an optional argument (to use for footnotes and the like).}
%\changes{v1.2}{2004/05/10}{Shared code moved to \cs{@openact}.}
%\changes{v1.2e}{2014/12/18}{Now \cs{act} actually uses \cs{actcontentsline}.}
%|\act| switches between |\@act| and |\@sact|; in the first case a
%line is added to the table of contents and an argument is assigned
%to |\actmark|. The actual task of printing the heading is left to
%|\m@ke@cthead|.
%    \begin{macrocode}
\newcommand\act{%
    \@openact
    \secdef\@act\@sact}
\newcommand\@act[1][]{%
    \phantomsection
    \addcontentsline{toc}{chapter}{\actcontentsline}
    \actmark{\actname\ \theact}
    \m@ke@cthead{#1}
    \@afterindentfalse
    \@afterheading}
\newcommand\@sact[1][]{%
    \m@ke@cthead{#1}
    \@afterindentfalse
    \@afterheading}
%    \end{macrocode}
%\end{macro}
%\begin{macro}{\Act}
%\changes{v1.2}{2004/05/10}{Introduced macro \cs{Act} for sections
%with a title.}
%\changes{v1.2e}{2014/12/18}{Now \cs{Act} actually uses \cs{actcontentsline}.}
%|\Act| is defined in the standard way for sectioning commands. For
%its starred version relies upon |\@sact|
%    \begin{macrocode}
\newcommand\Act{%
    \@openact
    \secdef\@Act\@sact}
\def\@Act[#1]#2{%
    \phantomsection
    \ifnum\c@secnumdepth>\m@ne
        \addcontentsline{toc}{chapter}{\actcontentsline~#1}
    \else
        \addcontentsline{toc}{chapter}{#1}
    \fi
    \actmark{\actname\ \theact\ #1}
    \m@ke@cthead{#2}
    \@afterindentfalse
    \@afterheading}
%    \end{macrocode}
%\end{macro}
%\begin{macro}{\m@ke@cthead}
%|\m@ke@cthead| actually prints the headings.
%    \begin{macrocode}
\newcommand\m@ke@cthead[1]{%
    \actheadstart
    {\parindent \z@
    \ifnum\c@secnumdepth>\m@ne
        \printactname \printsep \printactnum
    \fi
        \printacttitle{#1}
    \afteract}
}
%    \end{macrocode}
%\end{macro}
%\begin{macro}{\@openscene}
%According to the option loaded for line numbering, |\@openscene|
%performs the needed operations.
%    \begin{macrocode}
\newcommand\@openscene{%
    \stepcounter{scene}
    \if@lnps
        \setcounter{storelineno}{0}
        \if@poemscol
            \setcounter{storeprintlineindex}{0}
        \else
            \refstepcounter{storelineno}
        \fi
    \fi
}
%    \end{macrocode}
%\end{macro}
%\begin{macro}{\scene}
%\changes{v1.1}{2004/01/19}{\cs{scene} is now defined in a standard
%way and has an optional argument (to use for footnotes and the
%like).}
%\changes{v1.2}{2004/05/10}{Shared code moved to \cs{@penscene}.}
%\changes{v1.2e}{2014/12/18}{Now \cs{scene} actually uses \cs{scenecontentsline}.}
%|\scene| switches between |\@scene| and |\@sscene|; in the first case a
%line is added to the table of contents and an argument is assigned
%to |\scenemark|. The actual task of printing the heading is left to
%|\m@kescenehead|.
%    \begin{macrocode}
\newcommand\scene{%
    \@openscene
    \secdef\@scene\@sscene}
\newcommand\@scene[1][]{%
    \phantomsection
    \addcontentsline{toc}{section}{\scenecontentsline}
    \scenemark{\scenename\ \thescene}
    \m@kescenehead{#1}
    \@afterindentfalse
    \@afterheading}
\newcommand\@sscene[1][]{%
    \m@kescenehead{#1}
    \@afterindentfalse
    \@afterheading}
%    \end{macrocode}
%\end{macro}
%\begin{macro}{\Scene}
%\changes{v1.2}{2004/05/10}{Introduced macro \cs{Scene} for sections
%with a title.}
%\changes{v1.2e}{2014/12/18}{Now \cs{Scene} actually uses \cs{scenecontentsline}.}
%|\Scene| is defined in the standard way for sectioning commands. For
%its starred version relies upon |\@sscene|
%    \begin{macrocode}
\newcommand\Scene{%
    \@openscene
    \secdef\@Scene\@sscene}
\def\@Scene[#1]#2{%
    \phantomsection
    \ifnum\c@secnumdepth>\z@
        \addcontentsline{toc}{section}{\scenecontentsline~#1}
    \else
        \addcontentsline{toc}{section}{#1}
    \fi
    \scenemark{\scenename\ \thescene\ #1}
    \m@kescenehead{#2}
    \@afterindentfalse
    \@afterheading}
%    \end{macrocode}
%\end{macro}
%\begin{macro}{\m@kescenehead}
%|\m@kescenehead| actually prints the headings.
%    \begin{macrocode}
\newcommand\m@kescenehead[1]{%
    \sceneheadstart
    {\parindent \z@
    \ifnum\c@secnumdepth>\z@
        \printscenename \printsep \printscenenum
    \fi
    \printscenetitle{#1}
    \afterscene}
}
%    \end{macrocode}
%\end{macro}
%\subsection{Defining characters}
%
%\begin{macro}{\Character}
%The macro |\Character| performs three different tasks. First,
%it creates, being \meta{name} the third argument, the command
%|\|\meta{name}, for use in stage directions; in order to achieve
%this task it uses |\@namedef| (see the latex source).
%
%In second place it creates a |\|\meta{name}|speaks| command, used
%for printing the speaker's name. It uses, for this purpose a
%|\n@me@ppend@nddef| macro which is similar to |\@namedef|. A
%conditional |\if@drverse| produces different formatting for the verse
%and the prose environment.
%
%Finally, in third place, if the first optional argument is given
%and |\@xcharacter| is called, it creates an internal command,
%still using |\n@me@ppend@nddef|, in the form
%|\persona|\meta{count}, where \meta{count} is a counter expressed
%in roman lowercase numerals increasing by one every time
%|\Character| is called. This family of commands is used by
%|\DramPer| when it prints the list of the characters.
%\changes{v1.2}{2004/05/10}{The first argument of \cs{Character} has
%been made optional so that documents printed with previous
%versions are not compatible with the present and the future
%versions.}
%\changes{v1.2}{2004/05/10}{Added hook for inserting a delimiter
%after the character's name: \cs{speakdel}.}
%    \begin{macrocode}
\newcommand\Character{%
    \@ifnextchar[{\@xcharacter}{\@character}}
\def\@xcharacter[#1]#2#3{%
    \stepcounter{character}
    \@character{#2}{#3}
    \n@me@ppend@nddef{persona}{@\Roman{character}}{\castfont #1}
}
\def\@character#1#2{%
    \@namedef{#2}{{\namefont #1}\xspace}
    \n@me@ppend@nddef{#2}{\@ppendname}{%
        \if@drverse
            {\speakstab\speaksfont{#1}\speaksdel\par\nobreak\addvspace{-\parskip}}
        \else
            \item[#1\speaksdel]
        \fi}
}
%    \end{macrocode}
%    \begin{macrocode}
\newcommand{\n@me@ppend@nddef}[2]{%
    \expandafter\def\csname#1#2\endcsname}
\newcommand{\@ppendname}{speaks}
%    \end{macrocode}
%\end{macro}
%\begin{macro}{CharacterGroup}
%\changes{v1.2}{2004/05/10}{Added environment for characters groups
%in the \emph{dramatis person\ae} list.}
%This environment is used for groups of characters in the
%\emph{Dramatis Person\ae} list. The main idea is that each group
%of characters should be treated as a single |\persona|\meta{count}
%when called by |\DramPer|, while inside it should behave like
%|\DramPer| itself -- in this case |\dogrouplist|, which is
%identical in structure. The main feature is that every instance of
%\envname{CharacterGroup} defines an internal counter whose name
%depends by another counter -- namely: \textsf{character} -- and
%this is used by the correspondent `call' to |\dogrouplist|.
%    \begin{macrocode}
\newenvironment{CharacterGroup}[1]{%
    \stepcounter{character}
    \newcounter{g\Roman{character}}
        \grouplist{#1}
}{}
%    \end{macrocode}
%The name and the first specification of the characters, the big
%parentheses, and the common denomination are arranged in boxes
%whose length can be specified by the user by means of \emph{ad
%hoc} commands.
%    \begin{macrocode}
\newsavebox{\tbox}
\newcommand\grouplist[1]{%
    \global\n@me@ppend@nddef{persona}{@\Roman{character}}{%
    \begin{lrbox}{\tbox}
        \begin{minipage}[c]{\CharWidth}\raggedright
        \leftmargini=0pt
        \begin{list}{}{\itemsep=0pt}
            \dogrouplist
        \end{list}
        \end{minipage}
    \end{lrbox}
    \parbox{\CharWidth}{\usebox{\tbox}}%
    \parbox{\ParenWidth}{$\left.\rule{0pt}{\ht\tbox}\right\}$}
    \parbox{\CastWidth}{\castfont #1\strut}}
}
%    \end{macrocode}
%    \begin{macrocode}
\newcommand{\dogrouplist}{%
    \ifnum\value{g\Roman{temp}}>\value{gtemp}
        \stepcounter{gtemp}
        \item\@nameuse{gpersona@\Roman{temp}@\Roman{gtemp}}\strut
        \dogrouplist
    \fi
    \setcounter{gtemp}{0}
}
%    \end{macrocode}
%\end{macro}
%\begin{macro}{\GCharacter}
%\changes{v1.2}{2004/05/10}{Added macro for introducing a single
%character inside a group in the \emph{dramatist person\ae} list.}
%This is the version of |\Character| to be used inside a
%\envname{CharacterGroup} environment. In this case the first
%argument is, obviously, mandatory.
%    \begin{macrocode}
\newcommand\GCharacter[3]{
    \stepcounter{g\Roman{character}}
    \global\@namedef{#3}{{\namefont #2}\xspace}
    \global\n@me@ppend@nddef{#3}{\@ppendname}{%
        \if@drverse
            {\speakstab\speaksfont #2\speaksdel\par\nobreak\addvspace{-\parskip}}
        \else
            \item[#2\speaksdel]
        \fi}
    \global\n@me@ppend@nddef{gpersona@\Roman{character}}{%
        @\Roman{g\Roman{character}}}{\castfont #1}
}
%    \end{macrocode}
%\end{macro}
%\begin{macro}{\speaker}
%\changes{v1.1}{2004/01/19}{Command \cs{speaker} added.}
%This command is provided for defining characters which must not
%appear in the `Dramatis Person\ae' list and are not mentioned in
%stage directions.
%    \begin{macrocode}
\newcommand\speaker[1]{%
    \if@drverse
        {\speakstab\speaksfont #1\speaksdel\par\nobreak\addvspace{-\parskip}}
    \else
        \item[#1\speaksdel]
    \fi}
%    \end{macrocode}
%\end{macro}
%\begin{macro}{\DramPer}
%\changes{v1.1}{2004/01/19}{\cs{DramPer} is now defined in a standard
%way and has an optional argument (to use for footnotes and the
%like).}
%The macro |\DramPer| prints in the list of \emph{Dramatis
%Person\ae} the characters previously defined by the first argument
%of |\Character|. This is done via the |\dodramperlist| macro,
%which recursively calls the |\persona|\meta{count} commands and
%put them in the list defined by |\DramPer|.
%    \begin{macrocode}
\newcommand{\DramPer}{%
    \@ifundefined{if@openright}{\clearpage}{%
        \if@openright\cleardoublepage\else\clearpage\fi}
    \secdef\@dramper\@sdramper}
%    \end{macrocode}
%    \begin{macrocode}
\newcommand\@dramper[1][]{%
    \phantomsection
    \addcontentsline{toc}{chapter}{\casttitlename}
    \drampermark{\casttitlename}
    \m@kedramperhead{#1}}
%    \end{macrocode}
%    \begin{macrocode}
\newcommand\@sdramper[1][]{%
    \m@kedramperhead{#1}}
%    \end{macrocode}
%    \begin{macrocode}
\newcommand\m@kedramperhead[1]{
    \castheadstart
    {\printcasttitle #1
    \aftercasttitle}
    \begin{list}{}{\leftmargin=0pt \itemsep=0pt}
    \dodramperlist
    \end{list}
}
%    \end{macrocode}
%    \begin{macrocode}
\newcommand{\dodramperlist}{%
    \ifnum\value{character}>\value{temp}
        \stepcounter{temp}
        \item\@nameuse{persona@\Roman{temp}}\strut
        \dodramperlist
    \fi
}
%    \end{macrocode}
%\end{macro}
%\subsection{Stage direction}
%\begin{macro}{\direct}
%In the prose environment |\direct| merely encloses its argument in
%plain braces and emphasizes it; and has no starred version. In the
%verse environment things are a little more complicated, a
%|\parbox| is involved and I have to admit the result is not really
%perfect -- yet I found no better solution. The starred version
%must be used at the end of a stanza.
%\changes{v1.2}{2004/05/10}{Parbox length made customizable.}
%    \begin{macrocode}
\newcommand{\direct}{%
    \@ifstar\@sdirect\@direct}
%    \end{macrocode}
%    \begin{macrocode}
\newcommand{\@direct}[1]{%
    \if@drverse
        \vskip2\normallineskip
        \parbox[b]{\dirwidth}{\dirdelimiter{{\itshape #1}}}\@centercr
    \else
        \dirdelimiter{{\itshape #1}}\unskip
    \fi
}
%    \end{macrocode}
%    \begin{macrocode}
\newcommand{\@sdirect}[1]{%
    \if@drverse
        \vskip2\normallineskip
        \parbox[b]{\dirwidth}{\dirdelimiter{\itshape #1}}\\!
    \else
        \starrederror
    \fi
}
%    \end{macrocode}
%    \begin{macrocode}
\newcommand{\dirdelimiter}[1]{(#1)}
%    \end{macrocode}
%\end{macro}
%\begin{macro}{\StageDir}
%It's a very simple command |\StageDir|: it merely calls the
%\envname{stagedir} environment. No more talking of it.
%    \begin{macrocode}
\newcommand{\StageDir}[1]{%
    \begin{stagedir}
    #1
    \end{stagedir}
}
%    \end{macrocode}
%\end{macro}
%\begin{macro}{stagedir}
%The \envname{stagedir} environment calls by default the
%\envname{quote} environment, but can be redefined by the user to
%do everything by means of |\StageDirConf|. I use here |\em|
%instead of |\emph| in order to avoid strange indentations --
%thanks to Christian Ebert for having recognized and solved the
%problem.
%    \begin{macrocode}
\newenvironment{stagedir}{%
    \StageDirOpenSettings}{%
    \StageDirCloseSettings\global\@stagedirtrue}
\newcommand\StageDirOpenSettings{\begin{quote}\em}
\newcommand\StageDirCloseSettings{\end{quote}}
%    \end{macrocode}
%    \begin{macrocode}
\newcommand\StageDirConf[2]{%
    \renewcommand\StageDirOpenSettings{#1}
    \renewcommand\StageDirCloseSettings{#2}
}
%    \end{macrocode}
%\end{macro}
%\subsection{Configuration settings}
%    \begin{macrocode}
\newcommand\actcontentsline{\actname\ \theact}
\newcommand{\actnamefont}{\scshape\Large}
\newcommand{\actnumfont}{\actnamefont}
\newcommand{\acttitlefont}{\actnamefont}
\newcommand{\actname}{Act}
\newcommand{\printactname}{\centering\actnamefont \actname}
\newcommand{\printactnum}{\actnumfont \theact}
\newcommand{\printacttitle}[1]{\acttitlefont\ #1}
\newcommand\scenecontentsline{\scenename\ \thescene}
\newcommand{\scenenamefont}{\scshape\large}
\newcommand{\scenenumfont}{\scenenamefont}
\newcommand{\scenetitlefont}{\scenenamefont}
\newcommand{\scenename}{Scene}
\newcommand{\printscenename}{\centering\scenenamefont \scenename}
\newcommand{\printscenenum}{\scenenumfont \theact\intersep\thescene}
\newcommand{\printscenetitle}[1]{\scenetitlefont\ #1}
\newcommand{\intersep}{\ --\ }
\newcommand{\printsep}{\ }
\newcommand{\printcasttitle}{\centering\casttitlefont \casttitlename}
\newcommand{\casttitlefont}{\Large\scshape}
\newcommand{\casttitlename}{Dramatis Person\ae}
\newcommand{\castfont}{\normalfont}
\newcommand{\namefont}{\scshape}
\newcommand{\speaksfont}{\scshape}
\newcommand{\speaksdel}{}
\newlength{\CharWidth}
\setlength{\CharWidth}{.3\textwidth}
\newlength{\ParenWidth}
\setlength{\ParenWidth}{.05\textwidth}
\newlength{\CastWidth}
\setlength{\CastWidth}{.6\textwidth}
%    \end{macrocode}
%    \begin{macrocode}
\def\actheadstart{\vspace*{\beforeactskip}}
\def\afteract{\par\nobreak\vskip\afteractskip}
\def\sceneheadstart{\vspace*{\beforesceneskip}}
\def\afterscene{\par\nobreak\vskip\aftersceneskip}
\def\castheadstart{\vspace*{\beforecastskip}}
\def\aftercasttitle{\par\nobreak\vskip\aftercasttitleskip}
\newcommand{\speakstab}{\hspace{\speaksskip}}
\newlength{\beforeactskip}
\setlength{\beforeactskip}{\baselineskip}
\newlength{\afteractskip}
\setlength{\afteractskip}{\baselineskip}
\newlength{\beforesceneskip}
\setlength{\beforesceneskip}{0pt}
\newlength{\aftersceneskip}
\setlength{\aftersceneskip}{\baselineskip}
\newlength{\beforecastskip}
\setlength{\beforecastskip}{0pt}
\newlength{\aftercasttitleskip}
\setlength{\aftercasttitleskip}{0pt}
\newlength{\speaksskip}
\setlength{\speaksskip}{1em}
\newlength{\dirwidth}
\setlength{\dirwidth}{.6\textwidth}
%    \end{macrocode}
%Default settings for the \envname{drama} environment.
%    \begin{macrocode}
\newdimen\speakswidth
\speakswidth\z@
\newdimen\speaksindent
\speaksindent=-\leftmargin
\newdimen\speechskip
\speechskip\itemsep
\newdimen\Dparsep
\Dparsep\z@
\newdimen\Dlabelsep
\Dlabelsep\labelsep
\newcommand{\speakslabel}[1]{%
    \hspace\labelsep \speaksfont{#1}}
%    \end{macrocode}
%\subsection{Error messages handling}
%    \begin{macrocode}
\newcommand{\starrederror}{\PackageError{dramatist}{%
    The starred version of this command is not available under the
    option you have chosen}
    {You probably misspelled the command.^^J%
    Only the `verse' option supports a starred version of this
    command.}
}
\newcommand{\lnpwarning}[1]{The option `lnp#1' is meaningless outside the%
                            `drama*' environment}
\newcommand{\inputfilewarning}{\PackageWarningNoLine{dramatist}{^^J^^J%
********************************************************^^J%
* No Configuration file found, using default settings. *^^J%
********************************************************^^J%
}}
\newcommand{\foundfile}{\PackageWarningNoLine{dramatist}{^^J^^J%
*******************************************^^J%
* Using Configuration file dramatist.cfg. *^^J%
*******************************************^^J%
}}
%    \end{macrocode}
%\subsection{Local configuration file}
%The following code inputs the local configuration file
%|dramatist.cfg|.
%    \begin{macrocode}
\InputIfFileExists{dramatist.cfg}{\foundfile}{\inputfilewarning}
%</dramatist>
%    \end{macrocode}
%
%\appendix
%In Appendix are given the terms under which the package and his documentation are released.
%%\section{The GNU General Public License}
%
%\begin{center}
%{\parindent 0in
%
%{\scshape The GNU General Public License}
%
%Version 2, June 1991
%
%Copyright \copyright\ 1989, 1991 Free Software Foundation, Inc.
%
%\bigskip
%
%59 Temple Place - Suite 330, Boston, MA  02111-1307, USA
%
%\bigskip
%
%Everyone is permitted to copy and distribute verbatim copies of
%this license document, but changing it is not allowed. }
%\end{center}
%
%\begin{center}
%{\bf\large Preamble}
%\end{center}
%
%
%The licenses for most software are designed to take away your
%freedom to share and change it.  By contrast, the GNU General
%Public License is intended to guarantee your freedom to share and
%change free software---to make sure the software is free for all
%its users.  This General Public License applies to most of the
%Free Software Foundation's software and to any other program whose
%authors commit to using it.  (Some other Free Software Foundation
%software is covered by the GNU Library General Public License
%instead.)  You can apply it to your programs, too.
%
%When we speak of free software, we are referring to freedom, not
%price. Our General Public Licenses are designed to make sure that
%you have the freedom to distribute copies of free software (and
%charge for this service if you wish), that you receive source code
%or can get it if you want it, that you can change the software or
%use pieces of it in new free programs; and that you know you can
%do these things.
%
%To protect your rights, we need to make restrictions that forbid
%anyone to deny you these rights or to ask you to surrender the
%rights.  These restrictions translate to certain responsibilities
%for you if you distribute copies of the software, or if you modify
%it.
%
%For example, if you distribute copies of such a program, whether
%gratis or for a fee, you must give the recipients all the rights
%that you have.  You must make sure that they, too, receive or can
%get the source code.  And you must show them these terms so they
%know their rights.
%
%We protect your rights with two steps: (1) copyright the software,
%and (2) offer you this license which gives you legal permission to
%copy, distribute and/or modify the software.
%
%Also, for each author's protection and ours, we want to make
%certain that everyone understands that there is no warranty for
%this free software.  If the software is modified by someone else
%and passed on, we want its recipients to know that what they have
%is not the original, so that any problems introduced by others
%will not reflect on the original authors' reputations.
%
%Finally, any free program is threatened constantly by software
%patents. We wish to avoid the danger that redistributors of a free
%program will individually obtain patent licenses, in effect making
%the program proprietary.  To prevent this, we have made it clear
%that any patent must be licensed for everyone's free use or not
%licensed at all.
%
%The precise terms and conditions for copying, distribution and
%modification follow.
%
%\begin{center}
%{\Large \sc Terms and Conditions For Copying, Distribution and
%  Modification}
%\end{center}
%
%
%\begin{enumerate}
%
%\addtocounter{enumi}{-1}
%
%\item
%This License applies to any program or other work which contains a
%notice placed by the copyright holder saying it may be distributed
%under the terms of this General Public License.  The ``Program'',
%below, refers to any such program or work, and a ``work based on
%the Program'' means either the Program or any derivative work
%under copyright law: that is to say, a work containing the Program
%or a portion of it, either verbatim or with modifications and/or
%translated into another language.  (Hereinafter, translation is
%included without limitation in the term ``modification''.) Each
%licensee is addressed as ``you''.
%
%Activities other than copying, distribution and modification are
%not covered by this License; they are outside its scope.  The act
%of running the Program is not restricted, and the output from the
%Program is covered only if its contents constitute a work based on
%the Program (independent of having been made by running the
%Program). Whether that is true depends on what the Program does.
%
%\item You may copy and distribute verbatim copies of the Program's source
%  code as you receive it, in any medium, provided that you conspicuously
%  and appropriately publish on each copy an appropriate copyright notice
%  and disclaimer of warranty; keep intact all the notices that refer to
%  this License and to the absence of any warranty; and give any other
%  recipients of the Program a copy of this License along with the Program.
%
%You may charge a fee for the physical act of transferring a copy,
%and you may at your option offer warranty protection in exchange
%for a fee.
%
%\item
%
%You may modify your copy or copies of the Program or any portion
%of it, thus forming a work based on the Program, and copy and
%distribute such modifications or work under the terms of Section 1
%above, provided that you also meet all of these conditions:
%
%\begin{enumerate}
%
%\item
%
%You must cause the modified files to carry prominent notices
%stating that you changed the files and the date of any change.
%
%\item
%
%You must cause any work that you distribute or publish, that in
%whole or in part contains or is derived from the Program or any
%part thereof, to be licensed as a whole at no charge to all third
%parties under the terms of this License.
%
%\item
%If the modified program normally reads commands interactively when
%run, you must cause it, when started running for such interactive
%use in the most ordinary way, to print or display an announcement
%including an appropriate copyright notice and a notice that there
%is no warranty (or else, saying that you provide a warranty) and
%that users may redistribute the program under these conditions,
%and telling the user how to view a copy of this License.
%(Exception: if the Program itself is interactive but does not
%normally print such an announcement, your work based on the
%Program is not required to print an announcement.)
%
%\end{enumerate}
%
%
%These requirements apply to the modified work as a whole.  If
%identifiable sections of that work are not derived from the
%Program, and can be reasonably considered independent and separate
%works in themselves, then this License, and its terms, do not
%apply to those sections when you distribute them as separate
%works.  But when you distribute the same sections as part of a
%whole which is a work based on the Program, the distribution of
%the whole must be on the terms of this License, whose permissions
%for other licensees extend to the entire whole, and thus to each
%and every part regardless of who wrote it.
%
%Thus, it is not the intent of this section to claim rights or
%contest your rights to work written entirely by you; rather, the
%intent is to exercise the right to control the distribution of
%derivative or collective works based on the Program.
%
%In addition, mere aggregation of another work not based on the
%Program with the Program (or with a work based on the Program) on
%a volume of a storage or distribution medium does not bring the
%other work under the scope of this License.
%
%\item
%You may copy and distribute the Program (or a work based on it,
%under Section 2) in object code or executable form under the terms
%of Sections 1 and 2 above provided that you also do one of the
%following:
%
%\begin{enumerate}
%
%\item
%
%Accompany it with the complete corresponding machine-readable
%source code, which must be distributed under the terms of Sections
%1 and 2 above on a medium customarily used for software
%interchange; or,
%
%\item
%
%Accompany it with a written offer, valid for at least three years,
%to give any third party, for a charge no more than your cost of
%physically performing source distribution, a complete
%machine-readable copy of the corresponding source code, to be
%distributed under the terms of Sections 1 and 2 above on a medium
%customarily used for software interchange; or,
%
%\item
%
%Accompany it with the information you received as to the offer to
%distribute corresponding source code.  (This alternative is
%allowed only for noncommercial distribution and only if you
%received the program in object code or executable form with such
%an offer, in accord with Subsection b above.)
%
%\end{enumerate}
%
%
%The source code for a work means the preferred form of the work
%for making modifications to it.  For an executable work, complete
%source code means all the source code for all modules it contains,
%plus any associated interface definition files, plus the scripts
%used to control compilation and installation of the executable.
%However, as a special exception, the source code distributed need
%not include anything that is normally distributed (in either
%source or binary form) with the major components (compiler,
%kernel, and so on) of the operating system on which the executable
%runs, unless that component itself accompanies the executable.
%
%If distribution of executable or object code is made by offering
%access to copy from a designated place, then offering equivalent
%access to copy the source code from the same place counts as
%distribution of the source code, even though third parties are not
%compelled to copy the source along with the object code.
%
%\item
%You may not copy, modify, sublicense, or distribute the Program
%except as expressly provided under this License.  Any attempt
%otherwise to copy, modify, sublicense or distribute the Program is
%void, and will automatically terminate your rights under this
%License. However, parties who have received copies, or rights,
%from you under this License will not have their licenses
%terminated so long as such parties remain in full compliance.
%
%\item
%You are not required to accept this License, since you have not
%signed it.  However, nothing else grants you permission to modify
%or distribute the Program or its derivative works.  These actions
%are prohibited by law if you do not accept this License.
%Therefore, by modifying or distributing the Program (or any work
%based on the Program), you indicate your acceptance of this
%License to do so, and all its terms and conditions for copying,
%distributing or modifying the Program or works based on it.
%
%\item
%Each time you redistribute the Program (or any work based on the
%Program), the recipient automatically receives a license from the
%original licensor to copy, distribute or modify the Program
%subject to these terms and conditions.  You may not impose any
%further restrictions on the recipients' exercise of the rights
%granted herein. You are not responsible for enforcing compliance
%by third parties to this License.
%
%\item
%If, as a consequence of a court judgment or allegation of patent
%infringement or for any other reason (not limited to patent
%issues), conditions are imposed on you (whether by court order,
%agreement or otherwise) that contradict the conditions of this
%License, they do not excuse you from the conditions of this
%License.  If you cannot distribute so as to satisfy simultaneously
%your obligations under this License and any other pertinent
%obligations, then as a consequence you may not distribute the
%Program at all.  For example, if a patent license would not permit
%royalty-free redistribution of the Program by all those who
%receive copies directly or indirectly through you, then the only
%way you could satisfy both it and this License would be to refrain
%entirely from distribution of the Program.
%
%If any portion of this section is held invalid or unenforceable
%under any particular circumstance, the balance of the section is
%intended to apply and the section as a whole is intended to apply
%in other circumstances.
%
%It is not the purpose of this section to induce you to infringe
%any patents or other property right claims or to contest validity
%of any such claims; this section has the sole purpose of
%protecting the integrity of the free software distribution system,
%which is implemented by public license practices.  Many people
%have made generous contributions to the wide range of software
%distributed through that system in reliance on consistent
%application of that system; it is up to the author/donor to decide
%if he or she is willing to distribute software through any other
%system and a licensee cannot impose that choice.
%
%This section is intended to make thoroughly clear what is believed
%to be a consequence of the rest of this License.
%
%\item
%If the distribution and/or use of the Program is restricted in
%certain countries either by patents or by copyrighted interfaces,
%the original copyright holder who places the Program under this
%License may add an explicit geographical distribution limitation
%excluding those countries, so that distribution is permitted only
%in or among countries not thus excluded.  In such case, this
%License incorporates the limitation as if written in the body of
%this License.
%
%\item
%The Free Software Foundation may publish revised and/or new
%versions of the General Public License from time to time.  Such
%new versions will be similar in spirit to the present version, but
%may differ in detail to address new problems or concerns.
%
%Each version is given a distinguishing version number.  If the
%Program specifies a version number of this License which applies
%to it and ``any later version'', you have the option of following
%the terms and conditions either of that version or of any later
%version published by the Free Software Foundation.  If the Program
%does not specify a version number of this License, you may choose
%any version ever published by the Free Software Foundation.
%
%\item
%If you wish to incorporate parts of the Program into other free
%programs whose distribution conditions are different, write to the
%author to ask for permission.  For software which is copyrighted
%by the Free Software Foundation, write to the Free Software
%Foundation; we sometimes make exceptions for this.  Our decision
%will be guided by the two goals of preserving the free status of
%all derivatives of our free software and of promoting the sharing
%and reuse of software generally.
%
%\begin{center}
%{\Large\sc No Warranty }
%\end{center}
%
%\item
%{\sc Because the program is licensed free of charge, there is no
%warranty for the program, to the extent permitted by applicable
%law.  Except when otherwise stated in writing the copyright
%holders and/or other parties provide the program ``as is'' without
%warranty of any kind, either expressed or implied, including, but
%not limited to, the implied warranties of merchantability and
%fitness for a particular purpose.  The entire risk as to the
%quality and performance of the program is with you.  Should the
%program prove defective, you assume the cost of all necessary
%servicing, repair or correction.}
%
%\item
%{\sc In no event unless required by applicable law or agreed to in
%writing will any copyright holder, or any other party who may
%modify and/or redistribute the program as permitted above, be
%liable to you for damages, including any general, special,
%incidental or consequential damages arising out of the use or
%inability to use the program (including but not limited to loss of
%data or data being rendered inaccurate or losses sustained by you
%or third parties or a failure of the program to operate with any
%other programs), even if such holder or other party has been
%advised of the possibility of such damages.}
%
%\end{enumerate}
%
%
%\begin{center}
%{\Large\sc End of Terms and Conditions}
%\end{center}
%
%
%\pagebreak[2]
%
%\subsection*{Appendix: How to Apply These Terms to Your New Programs}
%
%If you develop a new program, and you want it to be of the
%greatest possible use to the public, the best way to achieve this
%is to make it free software which everyone can redistribute and
%change under these terms.
%
%  To do so, attach the following notices to the program.  It is safest to
%  attach them to the start of each source file to most effectively convey
%  the exclusion of warranty; and each file should have at least the
%  ``copyright'' line and a pointer to where the full notice is found.
%
%\begin{quote}
%one line to give the program's name and a brief idea of what it does. \\
%Copyright (C) yyyy  name of author \\
%
%This program is free software; you can redistribute it and/or
%modify it under the terms of the GNU General Public License as
%published by the Free Software Foundation; either version 2 of the
%License, or (at your option) any later version.
%
%This program is distributed in the hope that it will be useful,
%but WITHOUT ANY WARRANTY; without even the implied warranty of
%MERCHANTABILITY or FITNESS FOR A PARTICULAR PURPOSE.  See the GNU
%General Public License for more details.
%
%You should have received a copy of the GNU General Public License
%along with this program; if not, write to the Free Software
%Foundation, Inc., 59 Temple Place - Suite 330, Boston, MA
%02111-1307, USA.
%\end{quote}
%
%Also add information on how to contact you by electronic and paper
%mail.
%
%If the program is interactive, make it output a short notice like
%this when it starts in an interactive mode:
%
%\begin{quote}
%Gnomovision version 69, Copyright (C) yyyy  name of author \\
%Gnomovision comes with ABSOLUTELY NO WARRANTY; for details type `show w'. \\
%This is free software, and you are welcome to redistribute it
%under certain conditions; type `show c' for details.
%\end{quote}
%
%
%The hypothetical commands {\tt show w} and {\tt show c} should
%show the appropriate parts of the General Public License.  Of
%course, the commands you use may be called something other than
%{\tt show w} and {\tt show c}; they could even be mouse-clicks or
%menu items---whatever suits your program.
%
%You should also get your employer (if you work as a programmer) or
%your school, if any, to sign a ``copyright disclaimer'' for the
%program, if necessary.  Here is a sample; alter the names:
%
%\begin{quote}
%Yoyodyne, Inc., hereby disclaims all copyright interest in the program \\
%`Gnomovision' (which makes passes at compilers) written by James Hacker. \\
%
%signature of Ty Coon, 1 April 1989 \\
%Ty Coon, President of Vice
%\end{quote}
%
%
%This General Public License does not permit incorporating your
%program into proprietary programs.  If your program is a
%subroutine library, you may consider it more useful to permit
%linking proprietary applications with the library.  If this is
%what you want to do, use the GNU Library General Public License
%instead of this License.
%
%\section{GNU Free Documentation License}
%
%\begin{center}
%
%{\scshape GNU Free Documentation License}
%
%Version 1.2, November 2002
%
%Copyright \copyright 2000,2001,2002  Free Software Foundation, Inc.
%
%\bigskip
%
%59 Temple Place, Suite 330, Boston, MA  02111-1307  USA
%
%\bigskip
%
%Everyone is permitted to copy and distribute verbatim copies
%of this license document, but changing it is not allowed.
%\end{center}
%
%
%\begin{center}
%{\bf\large Preamble}
%\end{center}
%
%The purpose of this License is to make a manual, textbook, or other
%functional and useful document ``free'' in the sense of freedom: to
%assure everyone the effective freedom to copy and redistribute it,
%with or without modifying it, either commercially or noncommercially.
%Secondarily, this License preserves for the author and publisher a way
%to get credit for their work, while not being considered responsible
%for modifications made by others.
%
%This License is a kind of ``copyleft'', which means that derivative
%works of the document must themselves be free in the same sense.  It
%complements the GNU General Public License, which is a copyleft
%license designed for free software.
%
%We have designed this License in order to use it for manuals for free
%software, because free software needs free documentation: a free
%program should come with manuals providing the same freedoms that the
%software does.  But this License is not limited to software manuals;
%it can be used for any textual work, regardless of subject matter or
%whether it is published as a printed book.  We recommend this License
%principally for works whose purpose is instruction or reference.
%
%\begin{center}
%{\Large\bf 1. APPLICABILITY AND DEFINITIONS}
%\end{center}
%
%This License applies to any manual or other work, in any medium, that
%contains a notice placed by the copyright holder saying it can be
%distributed under the terms of this License.  Such a notice grants a
%world-wide, royalty-free license, unlimited in duration, to use that
%work under the conditions stated herein.  The \textbf{``Document''}, below,
%refers to any such manual or work.  Any member of the public is a
%licensee, and is addressed as \textbf{``you''}.  You accept the license if you
%copy, modify or distribute the work in a way requiring permission
%under copyright law.
%
%A \textbf{``Modified Version''} of the Document means any work containing the
%Document or a portion of it, either copied verbatim, or with
%modifications and/or translated into another language.
%
%A \textbf{``Secondary Section''} is a named appendix or a front-matter section of
%the Document that deals exclusively with the relationship of the
%publishers or authors of the Document to the Document's overall subject
%(or to related matters) and contains nothing that could fall directly
%within that overall subject.  (Thus, if the Document is in part a
%textbook of mathematics, a Secondary Section may not explain any
%mathematics.)  The relationship could be a matter of historical
%connection with the subject or with related matters, or of legal,
%commercial, philosophical, ethical or political position regarding
%them.
%
%The \textbf{``Invariant Sections''} are certain Secondary Sections whose titles
%are designated, as being those of Invariant Sections, in the notice
%that says that the Document is released under this License.  If a
%section does not fit the above definition of Secondary then it is not
%allowed to be designated as Invariant.  The Document may contain zero
%Invariant Sections.  If the Document does not identify any Invariant
%Sections then there are none.
%
%The \textbf{``Cover Texts''} are certain short passages of text that are listed,
%as Front-Cover Texts or Back-Cover Texts, in the notice that says that
%the Document is released under this License.  A Front-Cover Text may
%be at most 5 words, and a Back-Cover Text may be at most 25 words.
%
%A \textbf{``Transparent''} copy of the Document means a machine-readable copy,
%represented in a format whose specification is available to the
%general public, that is suitable for revising the document
%straightforwardly with generic text editors or (for images composed of
%pixels) generic paint programs or (for drawings) some widely available
%drawing editor, and that is suitable for input to text formatters or
%for automatic translation to a variety of formats suitable for input
%to text formatters.  A copy made in an otherwise Transparent file
%format whose markup, or absence of markup, has been arranged to thwart
%or discourage subsequent modification by readers is not Transparent.
%An image format is not Transparent if used for any substantial amount
%of text.  A copy that is not ``Transparent'' is called \textbf{``Opaque''}.
%
%Examples of suitable formats for Transparent copies include plain
%ASCII without markup, Texinfo input format, LaTeX input format, SGML
%or XML using a publicly available DTD, and standard-conforming simple
%HTML, PostScript or PDF designed for human modification.  Examples of
%transparent image formats include PNG, XCF and JPG.  Opaque formats
%include proprietary formats that can be read and edited only by
%proprietary word processors, SGML or XML for which the DTD and/or
%processing tools are not generally available, and the
%machine-generated HTML, PostScript or PDF produced by some word
%processors for output purposes only.
%
%The \textbf{``Title Page''} means, for a printed book, the title page itself,
%plus such following pages as are needed to hold, legibly, the material
%this License requires to appear in the title page.  For works in
%formats which do not have any title page as such, ``Title Page'' means
%the text near the most prominent appearance of the work's title,
%preceding the beginning of the body of the text.
%
%A section \textbf{``Entitled XYZ''} means a named subunit of the Document whose
%title either is precisely XYZ or contains XYZ in parentheses following
%text that translates XYZ in another language.  (Here XYZ stands for a
%specific section name mentioned below, such as \textbf{``Acknowledgements''},
%\textbf{``Dedications''}, \textbf{``Endorsements''}, or \textbf{``History''}.)
%To \textbf{``Preserve the Title''}
%of such a section when you modify the Document means that it remains a
%section ``Entitled XYZ'' according to this definition.
%
%The Document may include Warranty Disclaimers next to the notice which
%states that this License applies to the Document.  These Warranty
%Disclaimers are considered to be included by reference in this
%License, but only as regards disclaiming warranties: any other
%implication that these Warranty Disclaimers may have is void and has
%no effect on the meaning of this License.
%
%\begin{center}
%{\Large\bf 2. VERBATIM COPYING}
%\end{center}
%
%You may copy and distribute the Document in any medium, either
%commercially or noncommercially, provided that this License, the
%copyright notices, and the license notice saying this License applies
%to the Document are reproduced in all copies, and that you add no other
%conditions whatsoever to those of this License.  You may not use
%technical measures to obstruct or control the reading or further
%copying of the copies you make or distribute.  However, you may accept
%compensation in exchange for copies.  If you distribute a large enough
%number of copies you must also follow the conditions in section 3.
%
%You may also lend copies, under the same conditions stated above, and
%you may publicly display copies.
%
%\begin{center}
%{\Large\bf 3. COPYING IN QUANTITY}
%\end{center}
%
%If you publish printed copies (or copies in media that commonly have
%printed covers) of the Document, numbering more than 100, and the
%Document's license notice requires Cover Texts, you must enclose the
%copies in covers that carry, clearly and legibly, all these Cover
%Texts: Front-Cover Texts on the front cover, and Back-Cover Texts on
%the back cover.  Both covers must also clearly and legibly identify
%you as the publisher of these copies.  The front cover must present
%the full title with all words of the title equally prominent and
%visible.  You may add other material on the covers in addition.
%Copying with changes limited to the covers, as long as they preserve
%the title of the Document and satisfy these conditions, can be treated
%as verbatim copying in other respects.
%
%If the required texts for either cover are too voluminous to fit
%legibly, you should put the first ones listed (as many as fit
%reasonably) on the actual cover, and continue the rest onto adjacent
%pages.
%
%If you publish or distribute Opaque copies of the Document numbering
%more than 100, you must either include a machine-readable Transparent
%copy along with each Opaque copy, or state in or with each Opaque copy
%a computer-network location from which the general network-using
%public has access to download using public-standard network protocols
%a complete Transparent copy of the Document, free of added material.
%If you use the latter option, you must take reasonably prudent steps,
%when you begin distribution of Opaque copies in quantity, to ensure
%that this Transparent copy will remain thus accessible at the stated
%location until at least one year after the last time you distribute an
%Opaque copy (directly or through your agents or retailers) of that
%edition to the public.
%
%It is requested, but not required, that you contact the authors of the
%Document well before redistributing any large number of copies, to give
%them a chance to provide you with an updated version of the Document.
%
%\begin{center}
%{\Large\bf 4. MODIFICATIONS}
%\end{center}
%
%You may copy and distribute a Modified Version of the Document under
%the conditions of sections 2 and 3 above, provided that you release
%the Modified Version under precisely this License, with the Modified
%Version filling the role of the Document, thus licensing distribution
%and modification of the Modified Version to whoever possesses a copy
%of it.  In addition, you must do these things in the Modified Version:
%
%\begin{itemize}
%\item[A.]
%   Use in the Title Page (and on the covers, if any) a title distinct
%   from that of the Document, and from those of previous versions
%   (which should, if there were any, be listed in the History section
%   of the Document).  You may use the same title as a previous version
%   if the original publisher of that version gives permission.
%
%\item[B.]
%   List on the Title Page, as authors, one or more persons or entities
%   responsible for authorship of the modifications in the Modified
%   Version, together with at least five of the principal authors of the
%   Document (all of its principal authors, if it has fewer than five),
%   unless they release you from this requirement.
%
%\item[C.]
%   State on the Title page the name of the publisher of the
%   Modified Version, as the publisher.
%
%\item[D.]
%   Preserve all the copyright notices of the Document.
%
%\item[E.]
%   Add an appropriate copyright notice for your modifications
%   adjacent to the other copyright notices.
%
%\item[F.]
%   Include, immediately after the copyright notices, a license notice
%   giving the public permission to use the Modified Version under the
%   terms of this License, in the form shown in the Addendum below.
%
%\item[G.]
%   Preserve in that license notice the full lists of Invariant Sections
%   and required Cover Texts given in the Document's license notice.
%
%\item[H.]
%   Include an unaltered copy of this License.
%
%\item[I.]
%   Preserve the section Entitled ``History'', Preserve its Title, and add
%   to it an item stating at least the title, year, new authors, and
%   publisher of the Modified Version as given on the Title Page.  If
%   there is no section Entitled ``History'' in the Document, create one
%   stating the title, year, authors, and publisher of the Document as
%   given on its Title Page, then add an item describing the Modified
%   Version as stated in the previous sentence.
%
%\item[J.]
%   Preserve the network location, if any, given in the Document for
%   public access to a Transparent copy of the Document, and likewise
%   the network locations given in the Document for previous versions
%   it was based on.  These may be placed in the ``History'' section.
%   You may omit a network location for a work that was published at
%   least four years before the Document itself, or if the original
%   publisher of the version it refers to gives permission.
%
%\item[K.]
%   For any section Entitled ``Acknowledgements'' or ``Dedications'',
%   Preserve the Title of the section, and preserve in the section all
%   the substance and tone of each of the contributor acknowledgements
%   and/or dedications given therein.
%
%\item[L.]
%   Preserve all the Invariant Sections of the Document,
%   unaltered in their text and in their titles.  Section numbers
%   or the equivalent are not considered part of the section titles.
%
%\item[M.]
%   Delete any section Entitled ``Endorsements''.  Such a section
%   may not be included in the Modified Version.
%
%\item[N.]
%   Do not retitle any existing section to be Entitled ``Endorsements''
%   or to conflict in title with any Invariant Section.
%
%\item[O.]
%   Preserve any Warranty Disclaimers.
%\end{itemize}
%
%If the Modified Version includes new front-matter sections or
%appendices that qualify as Secondary Sections and contain no material
%copied from the Document, you may at your option designate some or all
%of these sections as invariant.  To do this, add their titles to the
%list of Invariant Sections in the Modified Version's license notice.
%These titles must be distinct from any other section titles.
%
%You may add a section Entitled ``Endorsements'', provided it contains
%nothing but endorsements of your Modified Version by various
%parties--for example, statements of peer review or that the text has
%been approved by an organization as the authoritative definition of a
%standard.
%
%You may add a passage of up to five words as a Front-Cover Text, and a
%passage of up to 25 words as a Back-Cover Text, to the end of the list
%of Cover Texts in the Modified Version.  Only one passage of
%Front-Cover Text and one of Back-Cover Text may be added by (or
%through arrangements made by) any one entity.  If the Document already
%includes a cover text for the same cover, previously added by you or
%by arrangement made by the same entity you are acting on behalf of,
%you may not add another; but you may replace the old one, on explicit
%permission from the previous publisher that added the old one.
%
%The author(s) and publisher(s) of the Document do not by this License
%give permission to use their names for publicity for or to assert or
%imply endorsement of any Modified Version.
%
%\begin{center}
%{\Large\bf 5. COMBINING DOCUMENTS}
%\end{center}
%
%You may combine the Document with other documents released under this
%License, under the terms defined in section 4 above for modified
%versions, provided that you include in the combination all of the
%Invariant Sections of all of the original documents, unmodified, and
%list them all as Invariant Sections of your combined work in its
%license notice, and that you preserve all their Warranty Disclaimers.
%
%The combined work need only contain one copy of this License, and
%multiple identical Invariant Sections may be replaced with a single
%copy.  If there are multiple Invariant Sections with the same name but
%different contents, make the title of each such section unique by
%adding at the end of it, in parentheses, the name of the original
%author or publisher of that section if known, or else a unique number.
%Make the same adjustment to the section titles in the list of
%Invariant Sections in the license notice of the combined work.
%
%In the combination, you must combine any sections Entitled ``History''
%in the various original documents, forming one section Entitled
%``History''; likewise combine any sections Entitled ``Acknowledgements'',
%and any sections Entitled ``Dedications''.  You must delete all sections
%Entitled ``Endorsements''.
%
%\begin{center}
%{\Large\bf 6. COLLECTIONS OF DOCUMENTS}
%\end{center}
%
%You may make a collection consisting of the Document and other documents
%released under this License, and replace the individual copies of this
%License in the various documents with a single copy that is included in
%the collection, provided that you follow the rules of this License for
%verbatim copying of each of the documents in all other respects.
%
%You may extract a single document from such a collection, and distribute
%it individually under this License, provided you insert a copy of this
%License into the extracted document, and follow this License in all
%other respects regarding verbatim copying of that document.
%
%\begin{center}
%{\Large\bf 7. AGGREGATION WITH INDEPENDENT WORKS}
%\end{center}
%
%A compilation of the Document or its derivatives with other separate
%and independent documents or works, in or on a volume of a storage or
%distribution medium, is called an ``aggregate'' if the copyright
%resulting from the compilation is not used to limit the legal rights
%of the compilation's users beyond what the individual works permit.
%When the Document is included in an aggregate, this License does not
%apply to the other works in the aggregate which are not themselves
%derivative works of the Document.
%
%If the Cover Text requirement of section 3 is applicable to these
%copies of the Document, then if the Document is less than one half of
%the entire aggregate, the Document's Cover Texts may be placed on
%covers that bracket the Document within the aggregate, or the
%electronic equivalent of covers if the Document is in electronic form.
%Otherwise they must appear on printed covers that bracket the whole
%aggregate.
%
%\begin{center}
%{\Large\bf 8. TRANSLATION}
%\end{center}
%
%Translation is considered a kind of modification, so you may
%distribute translations of the Document under the terms of section 4.
%Replacing Invariant Sections with translations requires special
%permission from their copyright holders, but you may include
%translations of some or all Invariant Sections in addition to the
%original versions of these Invariant Sections.  You may include a
%translation of this License, and all the license notices in the
%Document, and any Warranty Disclaimers, provided that you also include
%the original English version of this License and the original versions
%of those notices and disclaimers.  In case of a disagreement between
%the translation and the original version of this License or a notice
%or disclaimer, the original version will prevail.
%
%If a section in the Document is Entitled ``Acknowledgements'',
%``Dedications'', or ``History'', the requirement (section 4) to Preserve
%its Title (section 1) will typically require changing the actual
%title.
%
%\begin{center}
%{\Large\bf 9. TERMINATION}
%\end{center}
%
%You may not copy, modify, sublicense, or distribute the Document except
%as expressly provided for under this License.  Any other attempt to
%copy, modify, sublicense or distribute the Document is void, and will
%automatically terminate your rights under this License.  However,
%parties who have received copies, or rights, from you under this
%License will not have their licenses terminated so long as such
%parties remain in full compliance.
%
%\begin{center}
%{\Large\bf 10. FUTURE REVISIONS OF THIS LICENSE}
%\end{center}
%
%The Free Software Foundation may publish new, revised versions
%of the GNU Free Documentation License from time to time.  Such new
%versions will be similar in spirit to the present version, but may
%differ in detail to address new problems or concerns.  See
%http://www.gnu.org/copyleft/.
%
%Each version of the License is given a distinguishing version number.
%If the Document specifies that a particular numbered version of this
%License ``or any later version'' applies to it, you have the option of
%following the terms and conditions either of that specified version or
%of any later version that has been published (not as a draft) by the
%Free Software Foundation.  If the Document does not specify a version
%number of this License, you may choose any version ever published (not
%as a draft) by the Free Software Foundation.
%
%\begin{center}
%{\Large\bf ADDENDUM: How to use this License for your documents}
%\end{center}
%
%To use this License in a document you have written, include a copy of
%the License in the document and put the following copyright and
%license notices just after the title page:
%
%\bigskip
%\begin{quote}
%    Copyright \copyright  YEAR  YOUR NAME.
%    Permission is granted to copy, distribute and/or modify this document
%    under the terms of the GNU Free Documentation License, Version 1.2
%    or any later version published by the Free Software Foundation;
%    with no Invariant Sections, no Front-Cover Texts, and no Back-Cover Texts.
%    A copy of the license is included in the section entitled ``GNU
%    Free Documentation License''.
%\end{quote}
%\bigskip
%
%If you have Invariant Sections, Front-Cover Texts and Back-Cover Texts,
%replace the ``with\dots Texts.'' line with this:
%
%\bigskip
%\begin{quote}
%    with the Invariant Sections being LIST THEIR TITLES, with the
%    Front-Cover Texts being LIST, and with the Back-Cover Texts being LIST.
%\end{quote}
%\bigskip
%
%If you have Invariant Sections without Cover Texts, or some other
%combination of the three, merge those two alternatives to suit the
%situation.
%
%If your document contains nontrivial examples of program code, we
%recommend releasing these examples in parallel under your choice of
%free software license, such as the GNU General Public License,
%to permit their use in free software.
%
% \StopEventually{
% \begin{thebibliography}{10}
%    \bibitem{knuth} Donald Knuth.
%        \newblock \emph{The \TeX book},
%        \newblock Addison--Wesley, Reading, MA, 1996.
%    \bibitem{mem} Peter Wilson
%        \newblock \emph{The Memoir Class},
%        \newblock The Herries Press, Normandy Park, WA, 2001
%        \newblock (Available from CTAN, \texttt{macros/latex/contrib/supported/memoir})
%    \bibitem{ver} Peter Wilson
%        \newblock \emph{Typesetting simple verse with \LaTeX},
%        \newblock (Available from CTAN, \texttt{macros/latex/contrib/supported/verse})
% \end{thebibliography}
% \PrintChanges
% \PrintIndex
% }
%\iffalse
%<*schiller>
%\fi
%    \begin{macrocode}
%% This file shows some basic features of package `dramatist', when
%% typesetting a drama in prose. See the commented line for more
%% informations.
%%
%% The source for this example is taken from Schiller's `The
%% Robbers'.

\documentclass{book}
\usepackage{dramatist}

\pagestyle{plain}
%% Maybe you want acts and scenes print their marks in the headings.
%% The following lines should work.
%%\makeatletter
%%\def\ps@myheadings{%%
%%    \renewcommand\drampermark[1]{\markboth{##1}{}}
%%    \renewcommand\actmark[1]{\markboth{##1}{}}
%%    \renewcommand\scenemark[1]{\markright{##1}}
%%    \def\@oddfoot{\hfil\thepage\hfil}
%%    \def\@evenfoot{\hfil\thepage\hfil}
%%    \def\@evenhead{\hfil\scshape\leftmark\hfil}%%
%%    \def\@oddhead{\hfil\scshape\rightmark\hfil}%%
%%}
%%\makeatother
%%\pagestyle{myheadings}

%% We may change some parameters in the look of acts and scenes:
%%\renewcommand{\actnamefont}{\bfseries\Large}
%%\renewcommand{\theact}{\Roman{act}}
%%\renewcommand{\scenenamefont}{\bfseries\large}
%%\renewcommand{\thescene}{\Roman{scene}}

%% We may change some parameters in the look of characters:
%%\renewcommand{\castfont}{\bfseries}
%%\renewcommand{\speaksfont}{\itshape}
%%\renewcommand{\speaksdel}{.\\}
%%\renewcommand{\namefont}{\bfseries}

%% We may change some parameters in the look of stage directions:
%%\StageDirConf{\begin{center}\begin{minipage}{.7\textwidth}\bfseries\centering}{\end{minipage}\end{center}}
%%\renewcommand{\dirdelimiter}[1]{[#1]}

%% If you want the name of character printed centered above the
%% dialogue, you should comment out the following lines.
%%\Dlabelsep=0pt
%%\renewcommand{\speakslabel}[1]{%%
%%    \hbox to\textwidth{\hfill\speaksfont{#1}\hfill}}

\author{Friederich Schiller}
\title{The Robbers}
\date{}

\begin{document}
\begin{titlepage}
\maketitle
\end{titlepage}

\tableofcontents

%% We define some characters appearing in the play.
\Character[MAXIMILIAN, COUNT VON MOOR.]{Old Moor}{moor}
%% group of characters.
\begin{CharacterGroup}{his Sons.}
\GCharacter{CHARLES,}{Charles Von Moor}{cha}
\GCharacter{FRANCIS,}{Francis}{fran}
\end{CharacterGroup}
\Character[AMELIA VON EDELREICH, his Niece.]{Amelia}{amelia}
%% group of characters.
\begin{CharacterGroup}{Libertines, afterwards Banditti}
\GCharacter{SPIEGELBERG,}{Spiegelberg}{spie}
\GCharacter{SCHWEITZER,}{Schweitzer}{schwei}
\GCharacter{GRIMM,}{Grimm}{grim}
\GCharacter{RAZMANN,}{Razmann}{raz}
\GCharacter{SCHUFTERLE,}{Schufterle}{schuf}
\GCharacter{ROLLER,}{Roller}{rol}
\GCharacter{KOSINSKY,}{Kosinsky}{kos}
\GCharacter{SCHWARTZ,}{Schwartz}{schwa}
\end{CharacterGroup}
\Character[HERMANN, the natural son of a Nobleman.]{Hermann}{her}
\Character[DANIEL, an old Servant of Count von Moor.]{Daniel}{dan}
\Character[PASTOR MOSER.]{Pastor Moser}{pm}
\Character[FATHER DOMINIC, a Monk.]{Father Dominic}{fd}
%% the following collective character appears in the play as single
%% instances, so we don't need define commands and entries for it.
\Character[BAND OF ROBBERS, SERVANTS, ETC.]{}{}

%% We call the dramatis personae list.
\DramPer

\act

%% Schiller puts some general informations about location in the
%% scene heading. So, we must use uppercase version of `\scene'. On
%% the other hand we don't want this information to appear in
%% headers and table of contents; hence the empty optional argument.
\Scene[]{. -- Franconia}

\StageDir{Apartment in the Castle of COUNT MOOR.\\\fran, \moor}

\begin{drama}
\franspeaks But are you really well, father? You look so pale.

\moorspeaks Quite well, my son -- what have you to tell me?

\franspeaks The post is arrived -- a letter from our correspondent at
Leipsic.

\moorspeaks \direct{eagerly} Any tidings of my son Charles?

\franspeaks Hem! Hem! -- Why, yes. But I fear -- I know not -- whether I dare
 -- your health. -- Are you really quite well, father?

\moorspeaks As a fish in water. Does he write of my son? What means this
anxiety about my health? You have asked me that question twice.

\franspeaks If you are unwell -- or are the least apprehensive of being so --
permit me to defer -- I will speak to you at a fitter season. -- \direct{Half
aside.} These are no tidings for a feeble frame.

\moorspeaks Gracious Heavens? what am I doomed to hear?

\franspeaks First let me retire and shed a tear of compassion for my lost
brother. Would that my lips might be forever sealed -- for he is your
son! Would that I could throw an eternal veil over his shame -- for he is
my brother! But to obey you is my first, though painful, duty -- forgive
me, therefore.

\moorspeaks Oh, Charles! Charles! Didst thou but know what thorns thou
plantest in thy father's bosom! That one gladdening report of thee would
add ten years to my life! yes, bring back my youth! whilst now, alas,
each fresh intelligence but hurries me a step nearer to the grave!

\franspeaks Is it so, old man, then farewell! for even this very day we
might all have to tear our hair over your coffin.

\moorspeaks Stay! There remains but one short step more -- let him have his
will! \direct{He sits down.} The sins of the father shall be visited unto the
third and fourth generation -- let him fulfil the decree.

\franspeaks \direct{takes the letter out of his pocket}. You know our
correspondent! See! I would give a finger of my right hand might I
pronounce him a liar -- a base and slanderous liar! Compose yourself!
Forgive me if I do not let you read the letter yourself. You cannot,
must not, yet know all.

\moorspeaks All, all, my son. You will but spare me crutches.

\franspeaks \direct{reads} ``Leipsic, May 1. Were I not bound by an inviolable
promise to conceal nothing from you, not even the smallest particular,
that I am able to collect, respecting your brother's career, never, my
dearest friend, should my guiltless pen become an instrument of torture
to you. I can gather from a hundred of your letters how tidings such as
these must pierce your fraternal heart. It seems to me as though I saw
thee, for the sake of this worthless, this detestable'' -- \direct{\moor covers
his face}. Oh! my father, I am only reading you the mildest passages --
``this detestable man, shedding a thousand tears.'' Alas! mine flowed -- ay,
gushed in torrents over these pitying cheeks. ``I already picture to
myself your aged pious father, pale as death.'' Good Heavens! and so you
are, before you have heard anything.

\moorspeaks Go on! Go on!
\end{drama}

\begin{center}
[\dots]
\end{center}

\Scene[]{. -- A Tavern on the Frontier of Saxony.}

\StageDir{\cha intent on a book; \spie drinking at the table.}

\begin{drama}
\chaspeaks \direct{lays the book aside}. I am disgusted with this age of
puny scribblers when I read of great men in my Plutarch.

\spiespeaks \direct{places a glass before him, and drinks.} Josephus is the book
you should read.

\chaspeaks The glowing spark of Prometheus is burnt out, and now
they substitute for it the flash of lycopodium, a stage-fire which will
not so much as light a pipe. The present generation may be compared to
rats crawling about the club of Hercules.

A French abbe lays it down that Alexander was a poltroon; a phthisicky
professor, holding at every word a bottle of sal volatile to his nose,
lectures on strength. Fellows who faint at the veriest trifle criticise
the tactics of Hannibal; whimpering boys store themselves with phrases
out of the slaughter at Canna; and blubber over the victories of Scipio,
because they are obliged to construe them.

\spiespeaks Spouted in true Alexandrian style.

\chaspeaks A brilliant reward for your sweat in the battle-field
truly to have your existence perpetuated in gymnasiums, and your
immortality laboriously dragged about in a schoolboy's satchel. A
precious recompense for your lavished blood to be wrapped round
gingerbread by some Nuremberg chandler, or, if you have great luck, to
be screwed upon stilts by a French playwright, and be made to move on
wires! Ha, ha, ha!

\spiespeaks \direct{drinks.} Read Josephus, I tell you.

\chaspeaks Fie! fie upon this weak, effeminate age, fit for nothing
but to ponder over the deeds of former times, and torture the heroes of
antiquity with commentaries, or mangle them in tragedies. The vigor of
its loins is dried up, and the propagation of the human species has
become dependent on potations of malt liquor.
\end{drama}

\begin{center}
[\dots]
\end{center}

\act

\Scene[]{. -- {\scshape Francis von Moor} in his chamber -- in meditation.}

\begin{drama}
\franspeaks It lasts too long -- and the doctor even says is recovering -- an
old man's life is a very eternity! The course would be free and plain
before me, but for this troublesome, tough lump of flesh, which, like
the infernal demon-hound in ghost stories, bars the way to my treasures.

Must, then, my projects bend to the iron yoke of a mechanical system?
Is my soaring spirit to be chained down to the snail's pace of matter?
To blow out a wick which is already flickering upon its last drop of
oil -- 'tis nothing more. And yet I would rather not do it myself, on
account of what the world would say. I should not wish him to be
killed, but merely disposed of. I should like to do what your clever
physician does, only the reverse way -- not stop Nature's course by
running a bar across her path, but only help her to speed a little
faster. Are we not able to prolong the conditions of life? Why,
then, should we not also be able to shorten them? Philosophers and
physiologists teach us how close is the sympathy between the emotions of
the mind and the movements of the bodily machine. Convulsive sensations
are always accompanied by a disturbance of the mechanical vibrations --
passions injure the vital powers -- an overburdened spirit bursts its
shell. Well, then -- what if one knew how to smooth this unbeaten path,
for the easier entrance of death into the citadel of life? -- to work the
body's destruction through the mind -- ha! an original device! -- who can
accomplish this? -- a device without a parallel! Think upon it, Moor!
That were an art worthy of thee for its inventor. Has not poisoning
been raised almost to the rank of a regular science, and Nature
compelled, by the force of experiments, to define her limits, so that
one may now calculate the heart's throbbings for years in advance, and
say to the beating pulse, ``So far, and no farther''? Why should not one
try one's skill in this line?

And how, then, must I, too, go to work to dissever that sweet and
peaceful union of soul and body? What species of sensations should I
seek to produce? Which would most fiercely assail the condition of
life? Anger? -- that ravenous wolf is too quickly satiated. Care? that
worm gnaws far too slowly. Grief? -- that viper creeps too lazily for me.
Fear? -- hope destroys its power. What! and are these the only
executioners of man? is the armory of death so soon exhausted? \direct{In deep
thought.} How now! what! ho! I have it! \direct{Starting up.} Terror! What
is proof against terror? What powers have religion and reason under
that giant's icy grasp! And yet -- if he should withstand even this
assault? If he should! Oh, then, come Anguish to my aid! and thou,
gnawing Repentance! -- furies of hell, burrowing snakes who regorge your
food, and feed upon your own excrements; ye that are forever destroying,
and forever reproducing your poison! And thou, howling Remorse, that
desolatest thine own habitation, and feedest upon thy mother. And come
ye, too, gentle Graces, to my aid; even you, sweet smiling Memory,
goddess of the past -- and thou, with thy overflowing horn of plenty,
blooming Futurity; show him in your mirror the joys of Paradise, while
with fleeting foot you elude his eager grasp. Thus will I work my
battery of death, stroke after stroke, upon his fragile body, until the
troop of furies close upon him with Despair! Triumph! triumph! -- the
plan is complete -- difficult and masterly beyond compare -- sure -- safe; for
then \direct{with a sneer} the dissecting knife can find no trace of wound or
of corrosive poison.

\direct{Resolutely.} Be it so! \direct{Enter \her.} Ha! \emph{Deus ex machina}!
Hermann!

\herspeaks At your service, gracious sir!

\franspeaks \direct{shakes him by the hand.} You will not find it that of an
ungrateful master.

\herspeaks I have proofs of this.

\franspeaks And you shall have more soon -- very soon, Hermann! -- I have
something to say to thee, Hermann.

\herspeaks I am all attention.
\end{drama}

\begin{center}
[\dots]
\end{center}

\end{document}
%    \end{macrocode}
%\iffalse
%</schiller>
%\fi
%\iffalse
%<*vmarlowe>
%    \begin{macrocode}
%% This file shows some basic features of package `dramatist', when
%% typesetting a drama in verse. Here we use package `verse' as a
%% support for text in verse; if you usually use package `poemscol'
%% instead, you may give a look to file `marlowe-poemscol.tex'.
%%
%% The source for this example is taken from Marlowe's `Jew of
%% Malta'. I took leave to insert division in scenes, which is not
%% present in the original edition, in order to show more features.

\documentclass{book}
\usepackage{verse}
%% Warning: in this document I have used `\\!' (end of stanzas) to
%% mark the end of a speech, so that a space would be inserted
%% between speeches. This works only with the verse package or the
%% memoir class. Otherwise you must insert such a space by hand or
%% try to redefine \speakstab: e.g.
%% \renewcommand{\speakstab}{\bigskip\hspace{\speaksskip}}.

%% Load package `dramatist' after `verse'
\usepackage{dramatist}

%% comment out one of the following lines (and comment out the
%% previous one) if you want line numbering per act or per scene.
%%\usepackage[lnpa]{dramatist}
%%\usepackage[lnps]{dramatist}

\pagestyle{plain}
%% Maybe you want acts and scenes print their marks in the headings.
%% The following lines should work.
%%\makeatletter
%%\def\ps@myheadings{%%
%%    \renewcommand\drampermark[1]{\markboth{##1}{}}
%%    \renewcommand\actmark[1]{\markboth{##1}{}}
%%    \renewcommand\scenemark[1]{\markright{##1}}
%%    \def\@oddfoot{\hfil\thepage\hfil}
%%    \def\@evenfoot{\hfil\thepage\hfil}
%%    \def\@evenhead{\hfil\scshape\leftmark\hfil}%%
%%    \def\@oddhead{\hfil\scshape\rightmark\hfil}%%
%%}
%%\makeatother
%%\pagestyle{myheadings}

%% We may change some parameters in the look of acts and scenes:
%%\renewcommand{\actnamefont}{\bfseries\Large}
%%\renewcommand{\theact}{\Roman{act}}
%%\renewcommand{\scenenamefont}{\bfseries\large}
%%\renewcommand{\thescene}{\Roman{scene}}

%% We may change some parameters in the look of characters:
%%\renewcommand{\castfont}{\bfseries}
%%\renewcommand{\speaksfont}{\itshape}
%%\renewcommand{\speaksdel}{:}

%% We may change some parameters in the look of stage directions:
%%\StageDirConf{\begin{center}\begin{minipage}{.4\textwidth}\bfseries}{\end{minipage}\end{center}}

%% This is from verse package (or memoir class), for line numbering.
\poemlines{5}

\author{Christopher Marlowe}
\title{The Jew of Malta}
\date{}

\begin{document}
\begin{titlepage}
\maketitle
\end{titlepage}

\tableofcontents

%% We define some characters appearing in the play.
\Character[FERNEZE, governor of Malta.]{Ferneze}{fer}
\Character[LODOWICK, his son.]{Lodowick}{lod}
\Character[SELIM CALYMATH, son to the Grand Seignior.]{Calymath}{cal}
\Character[MARTIN DEL BOSCO, vice-admiral of Spain.]{Martin Del Bosco}{mar}
\Character[MATHIAS, a gentleman.]{Mathias}{mat}
%% group of characters.
\begin{CharacterGroup}{friars.}
\GCharacter{JACOMO,}{Jacomo}{jac}
\GCharacter{BARNARDINE,}{Barnardine}{barn}
\end{CharacterGroup}
\Character[BARABAS, a wealthy Jew.]{Barabas}{bar}
\Character[ITHAMORE, a slave.]{Ithamore}{ith}
\Character[PILIA-BORZA, a bully, attendant to BELLAMIRA.]{Pilia-Borza}{pb}
%% the following three collective characters appear in the play as
%% single instances, so we don't need define commands and entries
%% for them.
\Character[Two Merchants.]{}{}
\Character[Three Jews.]{}{}
\Character[Knights, Bassoes, Officers, Guard, Slaves, Messenger, and Carpenters]{}{}
\Character[KATHARINE, mother to MATHIAS.]{Katharine}{kat}
\Character[ABIGAIL, daughter to BARABAS.]{Abigail}{abi}
\Character[BELLAMIRA, a courtezan.]{Bellamira}{bel}
\Character[ABBESS.]{Abbess}{abb}
\Character[NUN.]{Nun}{nun}
%% This character doesn't appear in the dramatist personae list.
\Character{Machiavel}{mac}

%% We call the dramatis personae list.
\DramPer

%% The Prologue: we use \Act, but prevent it from printing \actname
%% and \theact.
\setcounter{secnumdepth}{-1}
\renewcommand{\printacttitle}[1]{\centering\acttitlefont #1}
%% This is needed only with myheadings.
%%\renewcommand{\actname}{}
%%\renewcommand{\theact}{}

\Act{Prologue}

\StageDir{Enter \mac.}

\begin{drama*}
\macspeaks
Albeit the world think Machiavel is dead,\\
Yet was his soul but flown beyond the Alps;\\
And, now the Guise is dead, is come from France,\\
To view this land, and frolic with his friends.\\
To some perhaps my name is odious;\\
But such as love me, guard me from their tongues,\\
And let them know that I am Machiavel,\\
And weigh not men, and therefore not men's words.\\
Admir'd I am of those that hate me most:\\
Though some speak openly against my books,\\
Yet will they read me, and thereby attain\\
To Peter's chair; and, when they cast me off,\\
Are poison'd by my climbing followers.\\
I count religion but a childish toy,\\
And hold there is no sin but ignorance.\\
Birds of the air will tell of murders past!\\
I am asham'd to hear such fooleries.\\
Many will talk of title to a crown:\\
What right had Caesar to the empery?\\
Might first made kings, and laws were then most sure\\
When, like the Draco's, they were writ in blood.\\
Hence comes it that a strong-built citadel\\
Commands much more than letters can import:\\
Which maxim had Phalaris observ'd,\\
H'ad never bellow'd, in a brazen bull,\\
Of great ones' envy:  o' the poor petty wights\\
Let me be envied and not pitied.\\
But whither am I bound?  I come not, I,\\
To read a lecture here in Britain,\\
But to present the tragedy of a Jew,\\
Who smiles to see how full his bags are cramm'd;\\
Which money was not got without my means.\\
I crave but this,--grace him as he deserves,\\
And let him not be entertain'd the worse\\
Because he favours me.\\
\direct*{Exit.}
%% The starred version of \direct can be used only at the end of a
%% speech, when the verse package or the memoir class has been
%% loaded.
\end{drama*}

%% Now we restore the default settings.
\setcounter{act}{0}
\setcounter{secnumdepth}{0}
\renewcommand{\printacttitle}[1]{\acttitlefont #1}
%% This is needed only with myheadings.
%%\renewcommand{\actname}{Act}
%%\renewcommand{\theact}{\roman{act}}
%%\renewcommand{\theact}{\Roman{act}}

%% Without a title, the command to be used is \act. We add a
%% footnote, just to show a feature.
\act[\footnote{This is the first act.}]

\scene
%% Scene are not marked in the original edition. I have introduced
%% them (somehow arbitrarily, I suppose) to give a complete review
%% of the available features.

\StageDir{\bar discovered in his counting-house, with heaps
of gold before him.}

\begin{drama*}
\barspeaks So that of thus much that return was made;\\
And of the third part of the Persian ships\\
There was the venture summ'd and satisfied.\\
As for those Samnites, and the men of Uz,\\
That bought my Spanish oils and wines of Greece,\\
Here have I purs'd their paltry silverlings.\\
Fie, what a trouble 'tis to count this trash!\\
Well fare the Arabians, who so richly pay\\
The things they traffic for with wedge of gold,\\
Whereof a man may easily in a day\\
Tell that which may maintain him all his life.\\
The needy groom, that never finger'd groat,\\
Would make a miracle of thus much coin;\\
But he whose steel-barr'd coffers are cramm'd full,\\
And all his life-time hath been tired,\\
Wearying his fingers' ends with telling it,\\
Would in his age be loath to labour so,\\
And for a pound to sweat himself to death.\\
Give me the merchants of the Indian mines,\\
That trade in metal of the purest mould;\\
The wealthy Moor, that in the eastern rocks\\
Without control can pick his riches up,\\
And in his house heap pearl like pebble-stones,\\
Receive them free, and sell them by the weight;\\
Bags of fiery opals, sapphires, amethysts,\\
Jacinths, hard topaz, grass-green emeralds,\\
Beauteous rubies, sparkling diamonds,\\
And seld-seen costly stones of so great price,\\
As one of them, indifferently rated,\\
And of a carat of this quantity,\\
May serve, in peril of calamity,\\
To ransom great kings from captivity.\\
This is the ware wherein consists my wealth;\\
And thus methinks should men of judgment frame\\
Their means of traffic from the vulgar trade,\\
And, as their wealth increaseth, so inclose\\
Infinite riches in a little room.\\
But now how stands the wind?\\
Into what corner peers my halcyon's bill?\\
Ha! to the east? yes.  See how stand the vanes--\\
East and by south:  why, then, I hope my ships\\
I sent for Egypt and the bordering isles\\
Are gotten up by Nilus' winding banks;\\
Mine argosy from Alexandria,\\
Loaden with spice and silks, now under sail,\\
Are smoothly gliding down by Candy-shore\\
To Malta, through our Mediterranean sea.--\\
But who comes here?\\
%% We need another character
\Character{Merchant}{mer}
\direct{Enter a \mer.}
How now!\\!

\merspeaks Barabas, thy ships are safe,\\
Riding in Malta-road; and all the merchants\\
With other merchandise are safe arriv'd,\\
And have sent me to know whether yourself\\
Will come and custom them.\\!

\barspeaks The ships are safe thou say'st, and richly fraught?\\!

\merspeaks They are.\\!

\barspeaks Why, then, go bid them come ashore,\\
And bring with them their bills of entry:\\
I hope our credit in the custom-house\\
Will serve as well as I were present there.\\
Go send 'em threescore camels, thirty mules,\\
And twenty waggons, to bring up the ware.\\
But art thou master in a ship of mine,\\
And is thy credit not enough for that?\\!
\end{drama*}

\begin{center}
[\dots]
\end{center}

\scene
%%Other characters.

\Character{knights}{knights}
\Character{officers}{officers}
\Character{bassoes}{bassoes}
\Character{First Basso}{fb}

%% An alternative formulation for \StageDir.
\begin{stagedir}
Enter \fer, governor of Malta, \knights, and \officers;
met by \cal, and \bassoes of the Turk.
\end{stagedir}

\begin{drama*}
\ferspeaks Now, bassoes, what demand you at our hands?\\!

\fbspeaks Know, knights of Malta, that we came from Rhodes,\\
From Cyprus, Candy, and those other isles\\
That lie betwixt the Mediterranean seas.\\!

\ferspeaks What's Cyprus, Candy, and those other isles\\
To us or Malta? what at our hands demand ye?\\!

\calspeaks The ten years' tribute that remains unpaid.\\!

\ferspeaks Alas, my lord, the sum is over-great!\\
I hope your highness will consider us.\\!

\calspeaks I wish, grave governor, 'twere in my power\\
To favour you; but 'tis my father's cause,\\
Wherein I may not, nay, I dare not dally.\\!
\end{drama*}

\begin{center}
[\dots]
\end{center}

\act

\scene

\StageDir{Enter \bar with a light.}

\begin{drama*}
\barspeaks Thus, like the sad-presaging raven, that tolls\\
The sick man's passport in her hollow beak,\\
And in the shadow of the silent night\\
Doth shake contagion from her sable wings,\\
Vex'd and tormented runs poor Barabas\\
With fatal curses towards these Christians.\\
The incertain pleasures of swift-footed time\\
Have ta'en their flight, and left me in despair;\\
And of my former riches rests no more\\
But bare remembrance; like a soldier's scar,\\
That has no further comfort for his maim.--\\
O Thou, that with a fiery pillar ledd'st\\
The sons of Israel through the dismal shades,\\
Light Abraham's offspring; and direct the hand\\
Of Abigail this night! or let the day\\
Turn to eternal darkness after this!--\\
No sleep can fasten on my watchful eyes,\\
Nor quiet enter my distemper'd thoughts,\\
Till I have answer of my Abigail.\\!
\end{drama*}

\StageDir{Enter \abi above.}

\begin{drama*}
\abispeaks Now have I happily espied a time\\
To search the plank my father did appoint;\\
And here, behold, unseen, where I have found\\
The gold, the pearls, and jewels, which he hid.\\!

\barspeaks Now I remember those old women's words,\\
Who in my wealth would tell me winter's tales,\\
And speak of spirits and ghosts that glide by night\\
About the place where treasure hath been hid:\\
And now methinks that I am one of those;\\
For, whilst I live, here lives my soul's sole hope,\\
And, when I die, here shall my spirit walk.\\!
\end{drama*}

\begin{center}
[\dots]
\end{center}

\end{document}
%    \end{macrocode}
%\fi
%\iffalse
%</vmarlowe>
%\fi
%\iffalse
%<*pmarlowe>
%\fi
%    \begin{macrocode}
%% This file shows some basic features of package `dramatist', when
%% typesetting a drama in verse. Here we use package `poemscol' as a
%% support for text in verse; if you usually use package `verse'
%% instead, you may give a look to file `marlowe-verse.tex'.
%%
%% The source for this example is taken from Marlowe's `Jew of
%% Malta'. I took leave to insert division in scenes, which is not
%% present in the original edition, in order to show more features.

\documentclass{book}
\usepackage{fancyhdr,newmarn,geometry}
\usepackage{keyval,ifthen,newcropmark}
\usepackage{poemscol}

%% Load package `dramatist' after `poemscol'
\usepackage{dramatist}

%% comment out one of the following lines (and comment out the
%% previous one) if you want line numbering per act or per scene.
%%\usepackage[lnpa]{dramatist}
%%\usepackage[lnps]{dramatist}


%% We may change some parameters in the look of acts and scenes:
%%\renewcommand{\actnamefont}{\bfseries\Large}
%%\renewcommand{\theact}{\Roman{act}}
%%\renewcommand{\scenenamefont}{\bfseries\large}
%%\renewcommand{\thescene}{\Roman{scene}}

%% We may change some parameters in the look of characters:
%%\renewcommand{\castfont}{\bfseries}
%%\renewcommand{\speaksfont}{\itshape}
%%\renewcommand{\speaksdel}{:}

%% We may change some parameters in the look of stage directions:
%%\StageDirConf{\begin{center}\begin{minipage}{.4\textwidth}\bfseries}{\end{minipage}\end{center}}

%% Comment out the following lines if you want to insert a vertical
%% space between speeches.
%%\makeatletter
%%\renewcommand{\speakstab}{\if@stagedir\global\@stagedirfalse\else\bigskip\fi\hspace{\speaksskip}}
%%\makeatother

%% Maybe you want acts and scenes print their marks in the headings.
%% The following lines should work.
%%\renewcommand{\drampermark}[1]{\volumeheader{#1}}
%%\renewcommand{\actmark}[1]{\leftheader{#1}}
%%\renewcommand{\scenemark}[1]{\volumeheader{#1}}
%%\fancyhf{}
%%\fancyhead[CO]{\scshape \volumeheadervalue}
%%\fancyhead[CE]{\scshape \leftheadervalue}
%%\fancyfoot[CE]{\thepage}
%%\fancyfoot[CO]{\thepage}

\author{Christopher Marlowe}
\title{The Jew of Malta}
\date{}

\begin{document}
\begin{titlepage}
\maketitle
\end{titlepage}

\tableofcontents

%% We define some characters appearing in the play.
\Character[FERNEZE, governor of Malta.]{Ferneze}{fer}
\Character[LODOWICK, his son.]{Lodowick}{lod}
\Character[SELIM CALYMATH, son to the Grand Seignior.]{Calymath}{cal}
\Character[MARTIN DEL BOSCO, vice-admiral of Spain.]{Martin Del Bosco}{mar}
\Character[MATHIAS, a gentleman.]{Mathias}{mat}
%% group of characters.
\begin{CharacterGroup}{friars.}
\GCharacter{JACOMO,}{Jacomo}{jac}
\GCharacter{BARNARDINE,}{Barnardine}{barn}
\end{CharacterGroup}
\Character[BARABAS, a wealthy Jew.]{Barabas}{bar}
\Character[ITHAMORE, a slave.]{Ithamore}{ith}
\Character[PILIA-BORZA, a bully, attendant to BELLAMIRA.]{Pilia-Borza}{pb}
%% the following three collective characters appear in the play as
%% single instances, so we don't need define commands and entries
%% for them.
\Character[Two Merchants.]{}{}
\Character[Three Jews.]{}{}
\Character[Knights, Bassoes, Officers, Guard, Slaves, Messenger, and Carpenters]{}{}
\Character[KATHARINE, mother to MATHIAS.]{Katharine}{kat}
\Character[ABIGAIL, daughter to BARABAS.]{Abigail}{abi}
\Character[BELLAMIRA, a courtezan.]{Bellamira}{bel}
\Character[ABBESS.]{Abbess}{abb}
\Character[NUN.]{Nun}{nun}
%% This character doesn't appear in the dramatist personae list.
\Character{Machiavel}{mac}

%% We call the dramatis personae list.
\DramPer

%% The Prologue: we use \Act, but prevent it from printing \actname
%% and \theact.
\setcounter{secnumdepth}{-1}
\renewcommand{\printacttitle}[1]{\centering\acttitlefont #1}
%% This is needed only with customized headings.
%%\renewcommand{\actname}{}
%%\renewcommand{\theact}{}
\Act{Prologue}

\StageDir{Enter \mac.}

\begin{drama*}
\macspeaks
Albeit the world think Machiavel is dead,\verseline
Yet was his soul but flown beyond the Alps;\verseline
And, now the Guise is dead, is come from France,\verseline
To view this land, and frolic with his friends.\verseline
To some perhaps my name is odious;\verseline
But such as love me, guard me from their tongues,\verseline
And let them know that I am Machiavel,\verseline
And weigh not men, and therefore not men's words.\verseline
Admir'd I am of those that hate me most:\verseline
Though some speak openly against my books,\verseline
Yet will they read me, and thereby attain\verseline
To Peter's chair; and, when they cast me off,\verseline
Are poison'd by my climbing followers.\verseline
I count religion but a childish toy,\verseline
And hold there is no sin but ignorance.\verseline
Birds of the air will tell of murders past!\verseline
I am asham'd to hear such fooleries.\verseline
Many will talk of title to a crown:\verseline
What right had Caesar to the empery?\verseline
Might first made kings, and laws were then most sure\verseline
When, like the Draco's, they were writ in blood.\verseline
Hence comes it that a strong-built citadel\verseline
Commands much more than letters can import:\verseline
Which maxim had Phalaris observ'd,\verseline
H'ad never bellow'd, in a brazen bull,\verseline
Of great ones' envy:  o' the poor petty wights\verseline
Let me be envied and not pitied.\verseline
But whither am I bound?  I come not, I,\verseline
To read a lecture here in Britain,\verseline
But to present the tragedy of a Jew,\verseline
Who smiles to see how full his bags are cramm'd;\verseline
Which money was not got without my means.\verseline
I crave but this,--grace him as he deserves,\verseline
And let him not be entertain'd the worse\verseline
Because he favours me.\verseline
\direct{Exit.}
\end{drama*}

%% Now we restore the default settings.
\setcounter{act}{0}
\setcounter{secnumdepth}{0}
\renewcommand{\printacttitle}[1]{\acttitlefont #1}
%% This is needed only with customized headings.
%%\renewcommand{\actname}{Act}
%%\renewcommand{\theact}{\roman{act}}
%%\renewcommand{\theact}{\Roman{act}}

%% Without a title, the command to be used is \act. We add a
%% footnote, just to show a feature.
\act[\footnote{This is the first act.}]

\scene
%% Scene are not marked in the original edition. I have introduced
%% them (somehow arbitrarily, I suppose) to give a complete review
%% of the available features.

\StageDir{\bar discovered in his counting-house, with heaps
of gold before him.}

\begin{drama*}
\barspeaks So that of thus much that return was made;\verseline
And of the third part of the Persian ships\verseline
There was the venture summ'd and satisfied.\verseline
As for those Samnites, and the men of Uz,\verseline
That bought my Spanish oils and wines of Greece,\verseline
Here have I purs'd their paltry silverlings.\verseline
Fie, what a trouble 'tis to count this trash!\verseline
Well fare the Arabians, who so richly pay\verseline
The things they traffic for with wedge of gold,\verseline
Whereof a man may easily in a day\verseline
Tell that which may maintain him all his life.\verseline
The needy groom, that never finger'd groat,\verseline
Would make a miracle of thus much coin;\verseline
But he whose steel-barr'd coffers are cramm'd full,\verseline
And all his life-time hath been tired,\verseline
Wearying his fingers' ends with telling it,\verseline
Would in his age be loath to labour so,\verseline
And for a pound to sweat himself to death.\verseline
Give me the merchants of the Indian mines,\verseline
That trade in metal of the purest mould;\verseline
The wealthy Moor, that in the eastern rocks\verseline
Without control can pick his riches up,\verseline
And in his house heap pearl like pebble-stones,\verseline
Receive them free, and sell them by the weight;\verseline
Bags of fiery opals, sapphires, amethysts,\verseline
Jacinths, hard topaz, grass-green emeralds,\verseline
Beauteous rubies, sparkling diamonds,\verseline
And seld-seen costly stones of so great price,\verseline
As one of them, indifferently rated,\verseline
And of a carat of this quantity,\verseline
May serve, in peril of calamity,\verseline
To ransom great kings from captivity.\verseline
This is the ware wherein consists my wealth;\verseline
And thus methinks should men of judgment frame\verseline
Their means of traffic from the vulgar trade,\verseline
And, as their wealth increaseth, so inclose\verseline
Infinite riches in a little room.\verseline
But now how stands the wind?\verseline
Into what corner peers my halcyon's bill?\verseline
Ha! to the east? yes.  See how stand the vanes--\verseline
East and by south:  why, then, I hope my ships\verseline
I sent for Egypt and the bordering isles\verseline
Are gotten up by Nilus' winding banks;\verseline
Mine argosy from Alexandria,\verseline
Loaden with spice and silks, now under sail,\verseline
Are smoothly gliding down by Candy-shore\verseline
To Malta, through our Mediterranean sea.--\verseline
But who comes here?\verseline
%% We need another character
\Character{Merchant}{mer}
\direct{Enter a \mer.}
How now!\verseline

\merspeaks Barabas, thy ships are safe,\verseline
Riding in Malta-road; and all the merchants\verseline
With other merchandise are safe arriv'd,\verseline
And have sent me to know whether yourself\verseline
Will come and custom them.\verseline

\barspeaks The ships are safe thou say'st, and richly fraught?\verseline

\merspeaks They are.\verseline

\barspeaks Why, then, go bid them come ashore,\verseline
And bring with them their bills of entry:\verseline
I hope our credit in the custom-house\verseline
Will serve as well as I were present there.\verseline
Go send 'em threescore camels, thirty mules,\verseline
And twenty waggons, to bring up the ware.\verseline
But art thou master in a ship of mine,\verseline
And is thy credit not enough for that?\verseline
\end{drama*}

\begin{center}
[\dots]
\end{center}

\scene
%%Other characters.

\Character{knights}{knights}
\Character{officers}{officers}
\Character{bassoes}{bassoes}
\Character{First Basso}{fb}

%% An alternative formulation for \StageDir.
\begin{stagedir}
Enter \fer, governor of Malta, \knights, and \officers;
met by \cal, and \bassoes of the Turk.
\end{stagedir}

\begin{drama*}
\ferspeaks Now, bassoes, what demand you at our hands?\verseline

\fbspeaks Know, knights of Malta, that we came from Rhodes,\verseline
From Cyprus, Candy, and those other isles\verseline
That lie betwixt the Mediterranean seas.\verseline

\ferspeaks What's Cyprus, Candy, and those other isles\verseline
To us or Malta? what at our hands demand ye?\verseline

\calspeaks The ten years' tribute that remains unpaid.\verseline

\ferspeaks Alas, my lord, the sum is over-great!\verseline
I hope your highness will consider us.\verseline

\calspeaks I wish, grave governor, 'twere in my power\verseline
To favour you; but 'tis my father's cause,\verseline
Wherein I may not, nay, I dare not dally.\verseline
\end{drama*}

\begin{center}
[\dots]
\end{center}

\act

\scene

\StageDir{Enter \bar with a light.}

\begin{drama*}
\barspeaks Thus, like the sad-presaging raven, that tolls\verseline
The sick man's passport in her hollow beak,\verseline
And in the shadow of the silent night\verseline
Doth shake contagion from her sable wings,\verseline
Vex'd and tormented runs poor Barabas\verseline
With fatal curses towards these Christians.\verseline
The incertain pleasures of swift-footed time\verseline
Have ta'en their flight, and left me in despair;\verseline
And of my former riches rests no more\verseline
But bare remembrance; like a soldier's scar,\verseline
That has no further comfort for his maim.--\verseline
O Thou, that with a fiery pillar ledd'st\verseline
The sons of Israel through the dismal shades,\verseline
Light Abraham's offspring; and direct the hand\verseline
Of Abigail this night! or let the day\verseline
Turn to eternal darkness after this!--\verseline
No sleep can fasten on my watchful eyes,\verseline
Nor quiet enter my distemper'd thoughts,\verseline
Till I have answer of my Abigail.\verseline
\end{drama*}

\StageDir{Enter \abi above.}

\begin{drama*}
\abispeaks Now have I happily espied a time\verseline
To search the plank my father did appoint;\verseline
And here, behold, unseen, where I have found\verseline
The gold, the pearls, and jewels, which he hid.\verseline

\barspeaks Now I remember those old women's words,\verseline
Who in my wealth would tell me winter's tales,\verseline
And speak of spirits and ghosts that glide by night\verseline
About the place where treasure hath been hid:\verseline
And now methinks that I am one of those;\verseline
For, whilst I live, here lives my soul's sole hope,\verseline
And, when I die, here shall my spirit walk.\verseline
\end{drama*}

\begin{center}
[\dots]
\end{center}

\end{document}
%    \end{macrocode}
%\iffalse
%</pmarlowe>
%\fi
%\Finale
% \PrintIndex \PrintChanges

\endinput
